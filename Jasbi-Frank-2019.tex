\documentclass[floatsintext,man]{apa6}

\usepackage{amssymb,amsmath}
\usepackage{ifxetex,ifluatex}
\usepackage{fixltx2e} % provides \textsubscript
\ifnum 0\ifxetex 1\fi\ifluatex 1\fi=0 % if pdftex
  \usepackage[T1]{fontenc}
  \usepackage[utf8]{inputenc}
\else % if luatex or xelatex
  \ifxetex
    \usepackage{mathspec}
    \usepackage{xltxtra,xunicode}
  \else
    \usepackage{fontspec}
  \fi
  \defaultfontfeatures{Mapping=tex-text,Scale=MatchLowercase}
  \newcommand{\euro}{€}
\fi
% use upquote if available, for straight quotes in verbatim environments
\IfFileExists{upquote.sty}{\usepackage{upquote}}{}
% use microtype if available
\IfFileExists{microtype.sty}{\usepackage{microtype}}{}

% Table formatting
\usepackage{longtable, booktabs}
\usepackage{lscape}
% \usepackage[counterclockwise]{rotating}   % Landscape page setup for large tables
\usepackage{multirow}		% Table styling
\usepackage{tabularx}		% Control Column width
\usepackage[flushleft]{threeparttable}	% Allows for three part tables with a specified notes section
\usepackage{threeparttablex}            % Lets threeparttable work with longtable

% Create new environments so endfloat can handle them
% \newenvironment{ltable}
%   {\begin{landscape}\begin{center}\begin{threeparttable}}
%   {\end{threeparttable}\end{center}\end{landscape}}

\newenvironment{lltable}
  {\begin{landscape}\begin{center}\begin{ThreePartTable}}
  {\end{ThreePartTable}\end{center}\end{landscape}}




% The following enables adjusting longtable caption width to table width
% Solution found at http://golatex.de/longtable-mit-caption-so-breit-wie-die-tabelle-t15767.html
\makeatletter
\newcommand\LastLTentrywidth{1em}
\newlength\longtablewidth
\setlength{\longtablewidth}{1in}
\newcommand\getlongtablewidth{%
 \begingroup
  \ifcsname LT@\roman{LT@tables}\endcsname
  \global\longtablewidth=0pt
  \renewcommand\LT@entry[2]{\global\advance\longtablewidth by ##2\relax\gdef\LastLTentrywidth{##2}}%
  \@nameuse{LT@\roman{LT@tables}}%
  \fi
\endgroup}


  \usepackage{graphicx}
  \makeatletter
  \def\maxwidth{\ifdim\Gin@nat@width>\linewidth\linewidth\else\Gin@nat@width\fi}
  \def\maxheight{\ifdim\Gin@nat@height>\textheight\textheight\else\Gin@nat@height\fi}
  \makeatother
  % Scale images if necessary, so that they will not overflow the page
  % margins by default, and it is still possible to overwrite the defaults
  % using explicit options in \includegraphics[width, height, ...]{}
  \setkeys{Gin}{width=\maxwidth,height=\maxheight,keepaspectratio}
\ifxetex
  \usepackage[setpagesize=false, % page size defined by xetex
              unicode=false, % unicode breaks when used with xetex
              xetex]{hyperref}
\else
  \usepackage[unicode=true]{hyperref}
\fi
\hypersetup{breaklinks=true,
            pdfauthor={},
            pdftitle={Adults' and Children's Comprehension of Disjunction in a Guessing Game: The role of measurement},
            colorlinks=true,
            citecolor=blue,
            urlcolor=blue,
            linkcolor=black,
            pdfborder={0 0 0}}
\urlstyle{same}  % don't use monospace font for urls

\setlength{\parindent}{0pt}
%\setlength{\parskip}{0pt plus 0pt minus 0pt}

\setlength{\emergencystretch}{3em}  % prevent overfull lines


% Manuscript styling
\captionsetup{font=singlespacing,justification=justified}
\usepackage{csquotes}
\usepackage{upgreek}

 % Line numbering
  \usepackage{lineno}
  \linenumbers


\usepackage{tikz} % Variable definition to generate author note

% fix for \tightlist problem in pandoc 1.14
\providecommand{\tightlist}{%
  \setlength{\itemsep}{0pt}\setlength{\parskip}{0pt}}

% Essential manuscript parts
  \title{Adults' and Children's Comprehension of Disjunction in a Guessing Game:
The role of measurement}

  \shorttitle{The Comprehension of Disjunction}


  \author{Masoud Jasbi\textsuperscript{1}~\& Michael C. Frank\textsuperscript{2}}

  % \def\affdep{{"", ""}}%
  % \def\affcity{{"", ""}}%

  \affiliation{
    \vspace{0.5cm}
          \textsuperscript{1} Harvard University\\
          \textsuperscript{2} Stanford University  }

  \authornote{
    All the experimental materials, data, randomization code, and analysis
    code for the studies reported in this paper are available in the
    following online repository:
    \href{https://github.com/jasbi/disjunction_comprehension}{https://github.com/jasbi/jasbi\_dissertation\_LearningDisjunction}.
    The repository also includes instructions for reproducing this research.
    
    Correspondence concerning this article should be addressed to Masoud
    Jasbi, Postal address. E-mail:
    \href{mailto:masoud_jasbi@fas.harvard.edu}{\nolinkurl{masoud\_jasbi@fas.harvard.edu}}
  }


  \abstract{Previous research suggests that adults and children might differ in
their interpretation of linguistic disjunction in two ways. First,
children might interpret \emph{or} as inclusive disjunction when adults
interpret it as exclusive (Crain, 2012). Second, unlike adults, children
might interpret a disjunction as logical conjunction (Singh, Wexler,
Astle-Rahim, Kamawar, \& Fox, 2016; Tieu et al., 2016). Here, we present
three studies that assess adults and children's understanding of
\emph{and} and \emph{or} using three different measures: binary
forced-choice judgments, ternary forced-choice judgments, and free-form
verbal feedback. Issues of measurement. Implications for pragmatic
development.}
  \keywords{conjunction, disjunction, implicatures, semantics, pragmatics, logical
connectives, language, acquisition, development, children \\

    \indent Word count: X
  }





\usepackage{amsthm}
\newtheorem{theorem}{Theorem}
\newtheorem{lemma}{Lemma}
\theoremstyle{definition}
\newtheorem{definition}{Definition}
\newtheorem{corollary}{Corollary}
\newtheorem{proposition}{Proposition}
\theoremstyle{definition}
\newtheorem{example}{Example}
\theoremstyle{definition}
\newtheorem{exercise}{Exercise}
\theoremstyle{remark}
\newtheorem*{remark}{Remark}
\newtheorem*{solution}{Solution}
\begin{document}

\maketitle

\setcounter{secnumdepth}{0}



\subsection{Introduction}\label{introduction}

Previous research has suggested that adults and children might differ in
their interpretation of \emph{or} in two ways. First, children might
interpret \emph{or} as inclusive disjunction when adults interpret it as
exclusive (Crain, 2012). Second, unlike adults, children might interpret
\emph{or} as logical conjunction, akin to \emph{and} (Singh et al.,
2016; Tieu et al., 2016). Here, we present three studies that assess
adults and children's understanding of \emph{and} and \emph{or} in a
guessing game paradigm. These studies show that four-year-olds'
interpretation of conjunction and disjunction may not be as different
from adults as previously supposed.

Study 1 tested adults' interpretations of logical connectives in the
context of a guessing game using Two and Three-Alternative Forced Choice
judgment tasks (2AFC and 3AFC). The results showed that adults interpret
\emph{and} and \emph{or} differently. They interpreted \emph{and} as
conjunction and \emph{or} as inclusive disjunction. However, in the task
with three alternatives (3AFC) adults did not consider a disjunction
felicitous when both disjuncts were true. Comparing the 2AFC and 3AFC
results, we find that the felicity of disjunctive statements is
sensitive to the measurement. 2AFC task systematically underestimated
judgments of felicity and better approximated truth judgments compared
to the 3AFC task. This finding is intuitive given that more options
provide a better opportunity to express nuances of linguistic
interpretation.

Study 2 investigated children's judgments in the same guessing game as
study 1 using a 3AFC task. We used three alternatives to give children a
better chance of expressing their pragmatic knowledge and judgments of
felicity (Katsos \& Bishop, 2011). The study also analyzed and
categorized children's open-ended spontaneous feedback to the guesser.
Both the 3AFC judgments and the categories of open-ended responses
showed that four-year-olds differentiated \emph{or} from \emph{and}.
While children's judgments in the 3AFC task showed no sign of infelicity
for disjunctive guesses when both disjuncts were true, their open-ended
feedback showed that children find such guesses infelicitous. In their
open-ended feedback, children's comments showed that use of a
conjunction in such cases would be more appropriate.

Study 3 used the same paradigm as study 2, but focused on replicating
children's open-ended responses and contrasting them with the results of
a 2AFC task. As in study 2, both truth judgments and open-ended feedback
showed that children differentiated \emph{or} from \emph{and}. The 2AFC
task showed no evidence that children find disjunctions with true
disjuncts infelicitous. However, children's judgments did not differ
significantly from those of adults in the 2AFC task of study 1. As in
study 2, children's open-ended feedback suggested that when both
disjuncts are true, children find a disjunctive statement infelicitous
and the conjunctive alternative more appropriate. Overall, the results
of study 2 and 3 show that forced-choice judgement tasks underestimate
children's pragmatic competence. Therefore, using open-ended elicitation
and analysis of children's feedback \textbf{along with} forced choice
judgment tasks may provide a better understanding of children's true
semantic and pragmatic knowledge.

The studies reported here build on previous studies, and fill two gaps
in the literature as well. First, most previous research focused on
children's interpretation of \emph{or} in complex sentences -- for
example with other logical words such as quantifiers \emph{every} and
\emph{none}. Here, we test children and adults' understanding of
\emph{and} and \emph{or} in simple existential sentences like
\enquote{\emph{There is a cat or a dog}.} To my knowledge, only Braine
and Rumain (1981) used simple existential constructions before, but
their experimental paradigm was relatively more complex than the
paradigm used here. As discussed before, simplifying the paradigm is an
important step in reducing conjunctive interpretations that arise due to
non-linguistic strategies. Second, most previous research tested
children and adults using 2AFC truth value judgment tasks (Crain \&
Thornton, 1998). Here, we report adults and children's judgments on both
2AFC and 3AFC tasks. We also use children's open-ended spontaneous
feedback to develop relevant analytical response categories and we
replicate the findings in a following pre-registered study. Katsos \&
Bishop (2011) argued that 3AFC judgment tasks are better suited for
assessing children's pragmatic competence. We present results that
suggest even a 3AFC task can underestimate children's pragmatic
knowledge and that children's spontaneous and open-ended elicited
responses provide valuable insights not available in forced choice
judgments.

\subsection{Previous Research}\label{previous-research}

Research on children's comprehension of logical connectives such as
\emph{and} and \emph{or} divides into two periods. The first period
(1960s-80s) was inspired by Piaget's developmental theory (Inhelder \&
Piaget, 1958). Researchers in this period sought to discover the
development of basic logical concepts such as negation, conjunction, and
disjunction. Following Inhelder and Piaget (1958), they predicted that
children first form concrete concepts for conjunction and disjunction
between the ages of 7-11 years (concrete operational stage) and only
after 11 (formal operational stage) do they develop an abstract and
logical understanding of these words. While later research in this
period rejected this timeline, it confirmed the idea that a logical
(inclusive) understanding of disjunction develops late. The second
period (since late 90s) is inspired by Grice's theory of meaning,
specifically his distinction between semantics and pragmatics.
Researchers in this period argue that previous studies conflated
semantic and pragmatic knowledge and used methods that vastly
underestimated children' semantic competence. By controlling for the
role of pramgatics and focusing on children's truth judgments, they show
that children have early and adult-like semantics for logical words such
as \emph{or}. Based on these results, they argue that the understanding
of logical concepts and their role in language is likely innate (Crain
\& Khlentzos, 2008, 2010). In what follows, I review the highlights of
these two traditions and end with a note on how the research presented
here contributes to this vast literature.

To examine the Piagetian theory, Nitta and Nagano (1966) tested 679
Japanese students (grades K, 2, 4, 6, and 8) and Neimark and Slotnick
(1970) conducted a similar study on 455 English-speaking children in
grades 3-8 and 58 college students. Participants were tested on
negation, conjunction, and disjunction. Each question provided six
response options; for example a fish, a bird, and a flower, each with a
white and a black version. Participants were asked to \enquote{circle
all the items} described by statements such as: \enquote{flower},
\enquote{not bird}, \enquote{bird and flower}, \enquote{bird or flower},
\enquote{black and bird}, \enquote{black or flower}, etc. These studies
concluded that the majority of the participants understood negation and
conjunction, but only college students correctly answered statements
containing a disjunction. They reported that participants made two types
of errors. First across all ages, some participants interpreted
disjunction as conjunction. For example they circled black birds when
the instruction said \enquote{black or bird}. Second, some selected only
one of the two categories. Based on these results Neimark (1970)
concluded that a \enquote{correct} (i.e.~inclusive) understanding of
disjunction only develops in the high school years and depends on the
attainment of formal operations as defined in the Piagetian theory.

Paris (1973) used a similar in-classroom setup to test children's
comprehension of connectives in Grades 2, 5, 8, 11, and college. Two
hundred participants (40 per grade) were asked to judge the truth of
sentences with the connectives \emph{and}, \emph{or}, and
\emph{either-or}. The experimenter showed participants slides of
pictures, for example a bird in a nest, with descriptions such as
\enquote{the bird is in the nest or the shoe is on the foot.} The
participants were asked to judge the statement as true or false. Paris
found that statements with \emph{and} were almost always judged
correctly, but this was not the case with disjunction. First, he
reported that older participants produced more errors when both
disjuncts were true, presumably because they interpreted disjunctions as
exclusive and not inclusive. Second, the majority of younger children,
and even around a fifth of college students considered a disjunction
false when only one of the disjuncts was true. The combination of these
two trends suggested that initially, children did not differentiate
\emph{or} from \emph{and}, interpreting both as conjunction. Finally,
Paris also found that there were fewer errors with \emph{either-or}
statements compared to \emph{or} statements. He suggested that the word
\emph{either} could provide further cue on how disjunction should be
interpreted. Paris (1973) attributed the conjunctive interpretations of
\emph{or} to children applying non-linguistic strategies when an
utterance is hard to interpret (See Clark, 1973 for a discussion of
nonlinguistic strategies in child language acquisition). He suggested
that children in his task were \enquote{comparing visual and auditory
information with little regard for the implied logical relationship in
the verbal description.} In other words, children responded with
\enquote{true} if the individual disjuncts matched the pictures and
false otherwise. Such a non-linguistic strategy would yield correct
answers for conjunction but incorrect (conjunctive) answers for
disjunction. This explains why conjunctive readings reduce with age and
why using the word \emph{either} helps reduce conjunctive
interpretations further.

It was understood that the in-class tests were not suitable for testing
participants' linguistic competence and certainly not suitable for
younger children. Therefore, Johansson and Sjolin (1975) set out to
examine the interpretaiton of disjunction in a simpler Give-item task.
They tested preschool Swedish-speaking children's comprehension of
conjunction and disjunction in present tense sentences (e.g.
\enquote{Richard wants to drink lemonade or milk. Show me what he
drank!}) and imperative sentences (e.g. \enquote{Put up the car or the
doll!}). They reported that starting (at least) at age four, children
interpreted the Swedish equivalents of \emph{and} and \emph{or} as
conjunction and exclusive disjunction. They argued that the linguistic
\emph{and} and \emph{or} should be kept separate from the logical
notions of conjunction and (inclusive) disjunction. While linguistic
understanding of \emph{and} and \emph{or} develops early and in
preschool years (as conjunction and exclusive disjunction), the logical
understanding of them develops late.

Braine and Rumain (1981) is the only study that tested the same
participants with both Give-item and Truth Value Judgment Tasks. They
tested 22 children in each of the age groups 5-6, 7-8, and 9-10 years,
as well as 22 adults. In the give-item task, 14 wooden blocks with
varying shapes, colors, and sizes were used (a replication of Suppes \&
Feldman, 1969). Experimenters asked participants the following: 1)
\enquote{Give me all the green things or give me all the round things}
and 2) \enquote{Give me all those things that are either blue or round.}
They reported that for both commands and in both children and adults,
the most likely response was to give all the objects that had only one
of the properties. They considered these results as evidence for a
\enquote{choose-one} (i.e.~exclusive) interpretation of disjunction in
the context of imperatives.

In the truth value judgment task, a puppet described the contents of
four boxes that each contained four animal toys. For example, the puppet
said \enquote{Either there is a horse or a duck in the box.} The first
box had both animals, the second had only a horse, the third only a
duck, and the last had neither. Participants were asked if the puppet
was right. The results showed that adults were split between an
inclusive and an exclusive interpretation of disjunction. The 7-8 and
9-10 year-olds were more likely to consider the disjunction as
inclusive. However, the youngest group (5-6 years old) was most likely
to interpret a disjunction similar to a conjunction: they said the
puppet was right when both animals were in the box and not right or
partly right if only one of the animals was in the box. Following Paris
(1973), Braine and Rumain (1981) argued that younger chidren do not take
the contribution of the connective \emph{or} into account. Instead, they
use a non-linguistic strategy in which the disjunction is right if both
propositions are true, partly right if only one is true, and wrong if
neither is true.

Braine and Rumain (1981) concluded that children's ability to interpret
a disjunction in a command develops earlier than their ability to judge
truth values. It is important to note that in Braine and Rumain (1981)'s
truth value judgment task, the puppet uses a disjunction even though the
content of the box was known to both the puppet and the participant
(lack of ignorance). Such uses of disjunction are infelicitous. More
generally, a disjunction such as \enquote{A or B} is infelicitious when
discourse participants already know which proposition is true. Later
truth value judgment studies such as Chierchia, Crain, Guasti, and
Thornton (1998) controlled for this effect of disjunction by making the
puppet utter disjunction as a prediction of an unknown event, and let
participants judge the prediction after they see the outcome of the
event.

Chierchia et al. (1998) kicked off the second period of inquiry into
children's comprehension of disjunciton. Following Grice (1989), they
differentiated between semantic knowledge, which includes the knowledge
of truth values, and pragmatic knowledge, which includes the knowldge
that conversational contributions ought to be truthful, informative,
relevant, and concise. They contended that interpreting logical
connectives involves a semantic and a pragmatic component, and that the
semantics of logical connectives cannot be assessed if the role of
pragmatics is not controlled for. More specifically, they argued that
felicitous use of a disjunction requires: (i) a set of alternatives (ii)
evidence that one of them holds (iiia) evdience that not all of them
hold, or (iiib) uncertainty as to whether all of them hold. While the
semantics of \emph{or} is inclusive, a variety of factors including
pragmatic reasoning can provide evidence that not all alternatives hold.
For example, we may reason that given speaker's knowledge of the
situation, she could have used the connective \emph{and} if all
alternative were true. Therefore, to understand the semantic
contribution of disjunction, we should test participants in contexts
which are stripped from pragmatic factors that contribute to
exclusivity.

They tested 23 English-speaking and 10 Italian-speaking children in two
conditions: description mode and prediction mode. In both conditions, a
troll considered whether to eat a hamburger, a piece of pizza, or an
ice-cream for lunch and went ahead to eat a piece of pizza and an
ice-cream but not a hamburger. In description mode, Kermit described
what happened as \enquote{A troll ate a piece of pizza or an ice cream}
while in prediction mode, Kermit used the same sentence as a prediction
before the troll eats his lunch. They reported that in the description
mode, children accepted Kermit's statement when both disjuncts were true
less than one-third of the time. However, in prediction mode, they
accepted such sentences 100\% of the time. They argued that when we
control for the effect of pragmatics on interpretation, children
understand disjunction as inclusive, and conform to the semantics of
disjunciton in classical logic.

Following Chierchia et al. (1998), several studies have argued that
preschool children's knowledge of disjunction conforms to the
predictions of classical logic and formal semantics in environments as
varied as negative sentences (Crain, Gualmini, \& Meroni, 2000),
conditional sentences (Gualmini, Crain, \& Meroni, 2000), restriction
and nuclear scope of the universal quantifeir \emph{every} (Chierchia,
Crain, Guasti, Gualmini, \& Meroni, 2001; Chierchia et al., 2004),
nuclear scope of the negative quantifier \emph{none} (Gualmini \& Crain,
2002), restriction and nuclear scope of \emph{not every} (Notley,
Thornton, \& Crain, 2012), and prepositional phrases headed by
\emph{before} (Notley, Zhou, Jensen, \& Crain, 2012), as well as similar
environments in other languages such as Mandarin Chinese and Japanese
(Goro \& Akiba, 2004; Su, 2014; Su \& Crain, 2013). These studies also
commonly reported that in linguistic environments where adults consider
a disjunction exclusive, children are more likely to consider it
inclusive. Since under the Gricean account, exclusive interpretation of
disjunction is the result of pragmatic (scalar) implicaatures, these
findings are considered as further evidence for the hypothesis that
young children do not compute implicatures at the rate that adults do
({\textbf{???}}; Noveck, 2001).

It is important to note that all the studies mentioned above in the
Gricean period use the Truth Value Judgment Task as specified in (Crain
\& Thornton, 1998). As mentioned earlier, Braine and Rumain (1981) found
that the same children were more likely to interpret a disjunction as
exclusvive in a give-item task and inclusive/conjunctive in a truth
value judgment task. Therefore, it is possible that truth value judgment
tasks are simply not suitable for capturing children' knowledge of
exclusivity implicatures. Furthermore, several studies listed above test
children's knowledge of disjunction in environments that largely
collapse the distinction between \emph{and} and \emph{or}. For example,
in the restriction of \emph{every}, a conjunction and a disjunction can
result in the same interpretation (e.g. \emph{Every man or woman is
happy} vs. \emph{Every man and woman is happy}). Therefore, successful
interpretation in these studies can also be achieved by applying the
nonlinguistic strategies that result in conjunctive interpretations, as
discussed by the early studies in the Piagetian period.

More recently, some studies have revived the earlier findings that
preschool children may interpret disjunction as conjunction. Singh et
al. (2016) tested 56 English-speaking children (M=4;11, 3;9-6;4) and 26
adults in a truth value judgment task. The experiment involved four
pictures: a boy holding a banana, a boy holding an apple and a banana,
three boys holding either an apple or a banana, and three boys holding
both apples and bananas. In each trial, participants saw one of the
pictures and a puppet described the pictures with four possible
utterances: \enquote{The/every boy is holding an apple or/and a banana.}
Participants were asked: \enquote{Was {[}the puppet{]} right or wrong
about this picture?} They found that children were more likely to say
the puppet was right when both disjuncts were true than when only one
was. They concluded that \enquote{many preschool children - the majority
in {[}the study's{]} sample - understand disjunctive sentences \ldots{}
as if they were conjunctions.}

Tieu et al. (2016) also found evidence for conjunctive interpretations
of disjunction in preschool children. They tested 28 French-speaking
children (3;7-6;6, M=4;5) and 18 Japanese-speaking children (4;7-6;6,
M=5;5) as well as 20 French-speaking and 21 Japanese-speaking adults.
They used the \enquote{prediction mode} of the Truth Value Judgment
Task, in which the puppet provides a prediction or guess, an event
occurs, and participants are asked if the prediction was right. For
example, there was a chicken on the screen, next to a toy bus and a toy
plane. The puppet appeared on the screen and predicted that \enquote{the
chicken pushed the bus or the plane.} Then the chicken pushed either one
or both of the objects. Participants stamped on a happy face or a sad
face to show whether the puppet's guess was right or wrong. Like Singh
et al. (2016), they reported that unlike adults, children were more
likely to consider the disjunctive guess right when both disjuncts were
true, rather than only one. They concluded that children - the majority
of them in their sample - interpreted disjunction as conjunction.

However, a recent replication of Tieu et al. (2016) by Skordos, Feiman,
Bale, and Barner (2018) suggests that the high rate of conjunctive
interpretations were most likely due to experimental design. They tested
126 preschoolers in three conditions: replication (N=43, 4;0-5;9,
M=5;0), modified script (N=41, 4;0-5;10, M=5;0), and three-alternatives
(N=42, 4;0-5;11, M=5;0). The first condition was a direct replication of
Tieu et al. (2016). The second, modified script, removed some
experimenter comments right after the puppet's guess that could
potentially confuse children. The comments were: \enquote{Look! The
chicken pushed that! She didn't want to break that one. So she didn't
touch it. So was {[}the puppet{]} right?} The third condition,
three-alternatives, was similar to modified-script but provided three
objects; for example a plane, a bus, and a bicycle. The reasoning was
that if there are only two alternatives, a disjunction is trivially
true, and consequently children may consider that unacceptable. The
results replicated Tieu et al. (2016)'s findings in the replication
condition, but showed that conjunctive interpretations of disjunction
disappeared almost completely in the third condition with three
alternatives. Skordos et al. (2018) concluded that children's
conjunctive interpretations are most likely due to non-linguistic
strategies applied when they are uncertain about some aspect of the
experimental task. This conclusion is similar to that of earlier studies
on conjunctive interpretations of disjunction in the 70s and 80s.

To summarize, previous studies show that the design of experimental
tasks can have a big impact on our conclusions regarding children's
comprehension of disjunction. Early in-class tasks suggested that even
high-schoolers do not interpret a disjunction correctly and confuse it
with \emph{and}. Improving on task design, Braine and Rumain (1981)
argued that this is only the case in preschool children. They also
showed that the same children can have different interpretations of
disjunction in different tasks: in a give-item task they interpret it as
exclusive while in a truth-value judgment task they interpret it as
conjunctive or inclusive. Using various versions of the truth value
judgment task, research in the Gricean tradition has argued that
preschool children understand the semantics of disjunciton and interpret
it as inclusive. However, this line of research has largely suggested
that children are insensitive to the exclusivity implicature of
disjunciton. While some recent studies have argued that preschool
children may interprete disjunction as conjunctive, a replication study
has argued that conjunctive interpretations were largely due to task
demands.

Here we improve on previous studies by first controling for various
factors that had proven problematic for previous studies, and second
investigating the role of measurement in preschool children's
interpretation of disjunction. As explained above, previous research has
shown that in studying children's interpretation of disjunction, it is
important to control for the following factors:

\begin{enumerate}
\def\labelenumi{\arabic{enumi}.}
\tightlist
\item
  complexity of the linguistic stimuli,
\item
  complexity of the task,
\item
  ignorance of the speaker with respect to the truth of the disjunts,
\item
  interpretation of the conjunction word (e.g. \emph{and}) in the same
  task
\item
  interpretation of adults in the same task.
\item
  Discernibility of conjunctive and disjunctive interpretations in the
  task
\end{enumerate}

Some previous studies used complex linguistic stimuli or relatively
complex designs that may have increased the application of
non-linguistic strategies. Some studies violated \enquote{speaker
ignorance}; i.e.~had the speaker utter the disjunction when the truth of
the propositions were known to the speaker. Some studies did not use the
conjunction word (e.g. \emph{and}) in control trials, or did not use
adults as control participants. Finally, some studies tested the
disjunction word in linguistic environments that collapse interpretive
differences between the conjunction and disjunction words. The
experimental paradigm reported here builds and improves on previous
studies by controlling for all these factors.

In the studies reported here, we used simple existential sentences (e.g.
\emph{there is a cat or a dog}) and tested the interpretation of
participants in a simple and easy to understand guessing game. The
guessing game provided a context in which the speaker was ignorance with
respect to to which alternatives actually hold. The game is essentially
a variant of the truth value judgment task. The study used conjunction
trials as well as adult participants as controls. The conjunction word
\emph{and} and the disjunction word \emph{or} resulted in different
interpretations in the task. Furthermore, we tested children's
interpretations in two different ways, using forced choice tasks with 2
and 3 options, as well as free form verbal responses.

\subsection{Study 1: Adult's 2AFC and 3AFC
Judgments}\label{study-1-adults-2afc-and-3afc-judgments}

The goal of this study was to examine adults' interpretations of
\emph{and} and \emph{or} as a benchmark for children's interpretations.
We designed the study as a guessing game. Participants saw a card, read
a description, and had to evaluate the description with respect to what
they saw on the card. In test trials, the descriptions contained the
conjunction word \emph{and} and the disjunction word \emph{or}. We
tested adults in both two-alternative and three-alternative forced
choice tasks (2AFC and 3AFC). The results suggested that adults
interpreted \emph{and} as conjunction and \emph{or} as inclusive
disjunction. Adults also considered statements with \emph{or}
infelicitous when both disjuncts were true. The study also found that
the 2AFC and 3AFC tasks registered different aspects of adult
interpretations: the 2AFC task captured adult intuitions on the basic
semantics of the connectives while the 3AFC task was sensitive to
pragmatic infelicities as well.

\subsubsection{Methods}\label{methods}

\paragraph{Materials and Design}\label{materials-and-design}

\begin{figure}[!h]

{\centering \includegraphics{figs/stimuli-1} 

}

\caption{Cards used in the connective guessing game.}\label{fig:stimuli}
\end{figure}

We used six cards with cartoon images of a cat, a dog, and an elephant
(Figure \ref{fig:stimuli}). There were two types of cards: cards with
only one animal and cards with two animals. There were three types of
guesses: simple (e.g. \emph{There is a cat}), conjunctive (e.g.
\emph{There is a cat and a dog}), and disjunctive (e.g. \emph{There is a
cat or a dog}). In each guess, the animal labels used in the guess and
the animal images on the card could have no overlap (e.g.~Image: dog,
Guess: \emph{There is a cat or an elephant}), partial overlap
(e.g.~Image: Cat, Guess: \emph{There is a cat or an elephant}), or total
overlap (e.g.~Image: cat and elephant, Guess: \emph{There is a cat or an
elephant}). Crossing the number of animals on the card, the types of
guesses, and the overlap between the guess and the card yields 12
different possible trial types. We chose 8 trial types (Figure
\ref{fig:trials}), to balance the number of one-animal vs.~two-animal
cards, simple vs.~connective guesses, and expected true vs.~false
trials.

\begin{figure}[!h]

{\centering \includegraphics{figs/trials-1} 

}

\caption{Trial types represented by example cards and example guesses.}\label{fig:trials}
\end{figure}

\paragraph{Participants and Procedure}\label{participants-and-procedure}

\begin{figure}[!h]

{\centering \includegraphics{figs/exampleTrial-1} 

}

\caption{An example trial in Study 1.}\label{fig:exampleTrial}
\end{figure}

We used Amazon's Mechanical Turk (MTurk) for recruitment and the online
platform Qualtrics for data collection and survey design. The task took
about 5 minutes on average to complete. 109 English speaking adults
participated. 57 of them were assigned to a 2AFC judgment task and 52 to
a 3AFC judgment task. In the 2AFC task, participants had to judge using
the options \enquote{wrong} and \enquote{right}. In the 3AFC task they
had to choose between \enquote{wrong}, \enquote{kinda right}, and
\enquote{right}. The two conditions were otherwise identical. There are
many possible labels for the middle option \enquote{kinda right},
including \enquote{kinda wrong} or \enquote{neither}. A later
experiment, tested different intermediate labels and found that adults
consider \enquote{kinda right} to be a more suitable option for
capturing pragmatic infelicities (see Jasbi, Waldon, \& Degen,
submitted). We expect similar behavior from labels like \enquote{a bit
right} and \enquote{a little right} which refer to non-maximal degrees
of being \enquote{right}.

The experiment had three phases: introduction, instruction, and test. In
the introduction, participants saw the six cards and read that they
would play a guessing game. Then a blindfolded cartoon character named
Bob appeared on the screen. Participants were told that in each round of
the game, they would see a card and Bob was going to guess what animal
was on the card. The study emphasized that Bob could not see anything.
Participants were asked to judge whether Bob's guess was right. In the
instruction phase, participants saw an example trial where a card with
the image of a dog was shown with the following sentence written above
Bob's head: \emph{There is a cat on the card}. All participants
correctly responded with \enquote{wrong} and proceeded to the test
phase.

In the test phase, participants saw one trial per trial type. Within
each trial type, the specific card-guess scenario was chosen at random.
The order of trial types was also randomized. At the end of the study,
participants received \$0.4 as compensation. Figure
\ref{fig:exampleTrial} shows an example test trial.

\begin{longtable}[]{@{}lllll@{}}
\caption{\label{tab:study1info}Summary of study 1 methods with adults
participants}\tabularnewline
\toprule
Study & N & Age & Mode & Response Options\tabularnewline
\midrule
\endfirsthead
\toprule
Study & N & Age & Mode & Response Options\tabularnewline
\midrule
\endhead
Study 1 - Part 1 & 57 & Adults & Online (Mturk) & Wrong,
Right\tabularnewline
Study 1 - Part 2 & 52 & Adults & Online (Mturk) & Wrong, Kinda Right,
Right\tabularnewline
\bottomrule
\end{longtable}

\subsubsection{Results}\label{results}

In this section, we first present the results of the 2AFC and 3AFC tasks
with adults. Then we discuss how these results can be interpreted with
respect to the semantics and pragmatics of disjunction in the context of
the guessing game.

\paragraph{Judgments with Two Alternatives
(2AFC)}\label{judgments-with-two-alternatives-2afc}

\begin{figure}
\centering
\includegraphics{figs/binaryAdultsPlot-1.pdf}
\caption{\label{fig:binaryAdultsPlot}Adults' two-alternative forced choice
judgments.}
\end{figure}

Figure \ref{fig:binaryAdultsPlot} shows the results for the adult 2AFC
task. The two left columns show the simple guesses and serve as
controls. The results show that if the animal mentioned in the guess was
not on the card (e.g., elephant), participants judged the guess to be
\enquote{wrong}; if the animal was on the card (e.g., cat), participants
judged the guess to be \enquote{right}. The next two columns of Figure
\ref{fig:binaryAdultsPlot} show the results for the test conditions,
namely conjunction and disjunction. An \emph{and}-guess (e.g.~cat and
dog) was considered \enquote{wrong} if only one of the animals was on
the card, and \enquote{right} if both were. An \emph{or}-guess (e.g.~cat
or dog) was \enquote{right} whether one or both animals were on the
card. The patterns of \enquote{right} and \enquote{wrong} responses in
the binary task match the expectations for truth and falsehood of
logical conjunction and (inclusive) disjunction.

\paragraph{Judgments with Three Alternatives
(3AFC)}\label{judgments-with-three-alternatives-3afc}

\begin{figure}
\centering
\includegraphics{figs/ternaryAdultsPlot-1.pdf}
\caption{\label{fig:ternaryAdultsPlot}Adults' three-alternative forced
choice judgments in the connective guessing game.}
\end{figure}

Figure \ref{fig:ternaryAdultsPlot} shows the results for the 3AFC
judgment task. For four trial types, the results were identical to the
2AFC task. In the first and second trial types, if the animal mentioned
was not on the card (e.g.~elephant), participants judged the guess as
\enquote{wrong}, regardless of wether one animal was on the card or two.
In the third trial type, if the animal mentioned (e.g.~cat) was the only
animal on the card, participants judged the guess as \enquote{right}.
Finally, if there were two animals on the card and the puppet mentioned
them using \emph{and} (e.g.~cat and dog), all participants considered
the guess \enquote{right}.

The four remaining trial types showed different patterns of judgments
than the ones in the 2AFC task. If the animal mentioned (e.g.~cat) was
only one of the animals on the card, participant judgments were divided
between \enquote{right} and \enquote{kinda right} (See Table
\ref{tab:binomStats}, row 1 for the statistical test). Also, most adults
considered a conjunctive guess (e.g.~cat and dog) \enquote{wrong}, when
only one of the animals was on the card (Table \ref{tab:binomStats}, row
2). However, some considered it \enquote{kinda right}, perhaps
suggesting that the intermediate option was used to express the notion
of partial truth. With respect disjunctive guesses (e.g.~cat or dog), if
the card had only one of the animals, most adults considers the guess
\enquote{right} while some considered it \enquote{kinda right} (Table
\ref{tab:binomStats}, row 3). It is possible that the adults who
considered such guesses \enquote{kinda right} were sensitive to the
under-informative nature of a disjunctive guess when a simple guess like
\enquote{cat} would have been more appropriate. If both animals were on
the card, adults were split between \enquote{kinda right} and
\enquote{right} responses (Table \ref{tab:binomStats}, row 4). The
choice of \enquote{kinda right} over \enquote{right} in such trials can
be interpreted as a sign that adults were sensitive to the infelicity of
a disjunction when conjunction was more appropriate. However, the scalar
reasoning with \emph{and} and \emph{or} is subtle and in section
\ref{implicature}, we discuss the nature of this reasoning in the
context of this guessing game.

\begin{longtable}[]{@{}lllllll@{}}
\caption{\label{tab:binomStats} Exact One-Sided Binomial
Test}\tabularnewline
\toprule
\begin{minipage}[b]{0.23\columnwidth}\raggedright\strut
Trial Type\strut
\end{minipage} & \begin{minipage}[b]{0.19\columnwidth}\raggedright\strut
\(n_{_{right}}/n_{_{total}}\)\strut
\end{minipage} & \begin{minipage}[b]{0.08\columnwidth}\raggedright\strut
\(\hat{p}_{_{right}}\)\strut
\end{minipage} & \begin{minipage}[b]{0.08\columnwidth}\raggedright\strut
\(p_{_{null}}\)\strut
\end{minipage} & \begin{minipage}[b]{0.08\columnwidth}\raggedright\strut
P-value\strut
\end{minipage} & \begin{minipage}[b]{0.08\columnwidth}\raggedright\strut
\(95\%~CI\)\strut
\end{minipage} & \begin{minipage}[b]{0.08\columnwidth}\raggedright\strut
\strut
\end{minipage}\tabularnewline
\midrule
\endfirsthead
\toprule
\begin{minipage}[b]{0.23\columnwidth}\raggedright\strut
Trial Type\strut
\end{minipage} & \begin{minipage}[b]{0.19\columnwidth}\raggedright\strut
\(n_{_{right}}/n_{_{total}}\)\strut
\end{minipage} & \begin{minipage}[b]{0.08\columnwidth}\raggedright\strut
\(\hat{p}_{_{right}}\)\strut
\end{minipage} & \begin{minipage}[b]{0.08\columnwidth}\raggedright\strut
\(p_{_{null}}\)\strut
\end{minipage} & \begin{minipage}[b]{0.08\columnwidth}\raggedright\strut
P-value\strut
\end{minipage} & \begin{minipage}[b]{0.08\columnwidth}\raggedright\strut
\(95\%~CI\)\strut
\end{minipage} & \begin{minipage}[b]{0.08\columnwidth}\raggedright\strut
\strut
\end{minipage}\tabularnewline
\midrule
\endhead
\begin{minipage}[t]{0.23\columnwidth}\raggedright\strut
Two Animals - Simple\strut
\end{minipage} & \begin{minipage}[t]{0.19\columnwidth}\raggedright\strut
32/52\strut
\end{minipage} & \begin{minipage}[t]{0.08\columnwidth}\raggedright\strut
0.62\strut
\end{minipage} & \begin{minipage}[t]{0.08\columnwidth}\raggedright\strut
0.50\strut
\end{minipage} & \begin{minipage}[t]{0.08\columnwidth}\raggedright\strut
0.06\strut
\end{minipage} & \begin{minipage}[t]{0.08\columnwidth}\raggedright\strut
0.49-1\strut
\end{minipage} & \begin{minipage}[t]{0.08\columnwidth}\raggedright\strut
\strut
\end{minipage}\tabularnewline
\begin{minipage}[t]{0.23\columnwidth}\raggedright\strut
One Animal - AND\strut
\end{minipage} & \begin{minipage}[t]{0.19\columnwidth}\raggedright\strut
16/52\strut
\end{minipage} & \begin{minipage}[t]{0.08\columnwidth}\raggedright\strut
0.69\strut
\end{minipage} & \begin{minipage}[t]{0.08\columnwidth}\raggedright\strut
0.50\strut
\end{minipage} & \begin{minipage}[t]{0.08\columnwidth}\raggedright\strut
0.00\strut
\end{minipage} & \begin{minipage}[t]{0.08\columnwidth}\raggedright\strut
0.57-1\strut
\end{minipage} & \begin{minipage}[t]{0.08\columnwidth}\raggedright\strut
\strut
\end{minipage}\tabularnewline
\begin{minipage}[t]{0.23\columnwidth}\raggedright\strut
One Animal - OR\strut
\end{minipage} & \begin{minipage}[t]{0.19\columnwidth}\raggedright\strut
19/52\strut
\end{minipage} & \begin{minipage}[t]{0.08\columnwidth}\raggedright\strut
0.63\strut
\end{minipage} & \begin{minipage}[t]{0.08\columnwidth}\raggedright\strut
0.50\strut
\end{minipage} & \begin{minipage}[t]{0.08\columnwidth}\raggedright\strut
0.04\strut
\end{minipage} & \begin{minipage}[t]{0.08\columnwidth}\raggedright\strut
0.51-1\strut
\end{minipage} & \begin{minipage}[t]{0.08\columnwidth}\raggedright\strut
\strut
\end{minipage}\tabularnewline
\begin{minipage}[t]{0.23\columnwidth}\raggedright\strut
Two Animals - OR\strut
\end{minipage} & \begin{minipage}[t]{0.19\columnwidth}\raggedright\strut
32/52\strut
\end{minipage} & \begin{minipage}[t]{0.08\columnwidth}\raggedright\strut
0.62\strut
\end{minipage} & \begin{minipage}[t]{0.08\columnwidth}\raggedright\strut
0.50\strut
\end{minipage} & \begin{minipage}[t]{0.08\columnwidth}\raggedright\strut
0.06\strut
\end{minipage} & \begin{minipage}[t]{0.08\columnwidth}\raggedright\strut
0.49-1\strut
\end{minipage} & \begin{minipage}[t]{0.08\columnwidth}\raggedright\strut
\strut
\end{minipage}\tabularnewline
\bottomrule
\end{longtable}

\subsubsection{Discussion}\label{discussion}

The example sentences bellow show the common interpretations of
conjunctive and disjunctive assertions (Aloni, 2016).

\begin{itemize}
\tightlist
\item
  Bob is sad \emph{and} angry.

  \begin{itemize}
  \tightlist
  \item
    Both are true. (Truth Conditional Meaning)
  \end{itemize}
\item
  Bob is sad \emph{or} angry.

  \begin{itemize}
  \tightlist
  \item
    At least one of the two is true. (Truth Conditional Meaning)
  \item
    Speaker doesn't know which is true. (Ignorance Inference)
  \item
    At most one of the two is true. (Exclusivity Inference)
  \end{itemize}
\end{itemize}

A conjunctive assertion implies that both propositions are true while a
disjunctive assertion implies that at least one is true. These two
inferences follow from the classical truth-conditional account of
conjunction and disjunction. They constitute the semantics of \emph{and}
and \emph{or}. However, a disjunctive assertion often has two additional
inferences: an ignorance inference and an exclusivity inference. These
additional inferences are often classified under pragmatic meaning. This
section discusses the semantics and pragmatics of \emph{and} and
\emph{or} in the context of the guessing game in Study 1.\footnote{See
  Gutzmann (2014) for a comprehensive discussion of the definitions and
  boundaries of semantics and pragmatics. Here my definitions and
  assumptions are close to those of Gazdar (1979).}

\paragraph{The Semantics of AND and
OR}\label{the-semantics-of-and-and-or}

Let's assume that the semantics of \emph{and} and \emph{or} in simple
declarative sentences like \enquote{there is a cat or(and) a dog} is
captured by the logical operators conjunction and inclusive disjunction
respectively. A conjunction is true when both conjuncts are true and
false otherwise. An inclusive disjunction is true when at least one
disjunct is true and false otherwise. Let's also assume a simple linking
function in which false statements are judged as \enquote{wrong} and
true statements as \enquote{right} (see Jasbi et al. (submitted) for a
discussion of linking assumptions in this task). In the context of study
1, this purely semantic (i.e.~truth-conditional) account has two main
predictions: 1. Conjunctive guesses like \enquote{cat and dog} are wrong
when only one of the animals is on the card. 2. Disjunctive guesses are
always right because in all such trials at least one of the animals is
present on the card. Figure \ref{fig:binaryAdultsPlot} shows that in
2AFC judgments, both predictions are borne out. In other words,
judgments with two alternatives seem to match the predictions of a
purely semantic account of the connectives \emph{and} and \emph{or} with
a linking function that considers \enquote{right} and \enquote{wrong}
roughly as \enquote{true} and \enquote{false}.

However, in the 3AFC task, judgments deviated from a purely semantic
account in four trial types: 1. disjunction trials with one animal 2.
disjunction trials with two animals, 3. conjunction trials with one
animal, and 4. trials with simple guesses when two animals were shown on
the card. Participants often used the third option \enquote{kinda right}
in these trial types. Other trial types obtained identical results in
2AFC and 3AFC tasks. The comparison of forced choice judgments with two
and three alternatives suggests that two alternatives better captured
the truth-conditional meaning of the connectives, but underestimated
adult pragmatic reasoning in the guessing game.

\paragraph{The Pragmatics of AND and OR}\label{implicature}

\begin{figure}
\centering
\includegraphics{figs/exclusivity-1.pdf}
\caption{\label{fig:exclusivity}Example of cards referred to by a
conjunction, inclusive disjunction, and exclusive disjunction.}
\end{figure}

A disjunctive assertion like \enquote{cat or dog} gives rise to an
ignorance inference and an exclusivity inference. The ignorance
inference is the inference that the speaker does not know which disjunct
actually holds. For example in figure \ref{fig:exclusivity}, the
disjunctive guess is uncertain between three outcomes: cards 1, 2, and
3. A disjunction is infelicitous when the outcome is known to discourse
participants. For example, Tarski (1941) mentioned that a disjunction
like \enquote{the grass is green or blue} is odd because we already know
that the grass is green. The guessing game in this study controls for
this ignorance effect by keeping the guesser blindfolded. Therefore, all
the disjunctive guesses are evaluated in a context where participants
know that the guesser is ignorant of the animals on the cards - both the
number of them on the card and their identity. The exclusivity inference
is the inference that only one of the disjuncts holds and \textbf{not
both}. In figure \ref{fig:exclusivity}, a disjunction like \enquote{cat
or dog} only refers to cards 2 and 3 if it is accompanied by an
exclusivity inference.

Since Grice (1989), this exclusive interpretation of \emph{or} has been
(at least partly) attributed to pragmatic reasoning about the speaker's
connective choice. The reasoning goes like this: conversational
participants are required to make their utterances as informative as
possible. In the context of making predictions and guessing, a guesser
is required to make any guess as specific (i.e.~informative) as
possible.\footnote{When you ask someone to predict the outcome of a coin
  toss, a guess like \enquote{it will be heads or tails} does not count
  as a felicitous guess or prediction, presumably because it is not
  informative, ie.e. it will always be true.} A conjunction is more
specific and informative than a disjunction (Horn, 1989). For example in
Figure \ref{fig:exclusivity}, \emph{cat and dog} picks card 1 while
\emph{cat or dog} refers to cards 1, 2, and 3. If speakers intend to
refer to card 1, they should use \emph{and} and say \emph{cat and dog}.
If they use \emph{or} instead of \emph{and}, they probably do not intend
to refer to card 1. Following this line of reasoning, we can exclude the
possibility that a speaker intends to refer to card 1. The term
\enquote{exclusivity implicature} captures this pragmatic reasoning that
results in excluding the possibility of both disjuncts being true.

Our goal here is to lay out the structure of the exclusivity reasoning
in the experimental setup and explain how it may be manifested in the
results of the experimental studies. There are three main components to
the pragmatic reasoning in the guessing game: 1. the assumptions of the
game. 2. sensitivity to (under)informativity, and 3. the pragmatic
reasoning about the speaker's choice of connectives. Like Katsos \&
Bishop (2011), we have considered \enquote{sensitivity to
informativeness} as a precondition for \enquote{derivation of scalar
implicatures}. We begin with the assumptions of the guessing game.

\begin{itemize}
\tightlist
\item
  \textbf{Guessing Game Assumptions}:

  \begin{itemize}
  \tightlist
  \item
    \textbf{Ignorance}: the guesser does not know the number or identity
    of the animals on the card.
  \item
    \textbf{Specificity}: the guesser is required to be as specific as
    possible, ideally referring to a single card.
  \end{itemize}
\end{itemize}

As explained before, ignorance of the guesser was explicit and part of
the instructions in the study. However, specificity was an implicit
assumption\footnote{Making this assumption explicit is both hard for
  young children and almost impossible when disjunctive guesses are
  used. Disjunctive guesses are always underinformative and never pick
  out a specific card.}. All the guesses used in the experiment can pick
a single card except for disjunctive ones. Conjunctive guesses like
\emph{cat and dog} pick specific cards. The simple ones like \emph{cat}
can be strengthened pragmatically to mean \enquote{only a cat}, and pick
a specific card. However, Disjunctive ones like \emph{cat or dog} pick
two cards in their most specific (exclusive) sense. Therefore, they are
always under-informative and violate the specificity assumption.

\begin{itemize}
\item
  \textbf{Sensitivity to Informativeness}: The guesser said \emph{cat or
  dog} which is under-informative and picks cards 1, 2, and 3.
\item
  \textbf{Violation Assumption}: the guesser is violating the
  specificity requirement.
\end{itemize}

Participants can detect the underinformativity of disjunctive guesses,
notice the violation of specificity, and then decide whether they would
like to tolerate this violation or punish it. It should be pointed out
that it is hard to distinguish between \enquote{tolerating the
specificity violation} and simply revising the specificity assumption of
the game to avoid a violation. For example, participants may assume that
the goal of the game is saying something true about the cards rather
than being as specific as possible. In either case, the prediction is
that adults who tolerate violation or revise specificity would judge
disjunctive guesses as \enquote{right}. However, if participants assume
specificity and decide to not tolerate its violation, they will judge
all disjunctive guesses to have some degree of infelicity. Since an
under-informative guess is still technically correct, participants may
not punish such a guess with a \enquote{wrong} response and prefer an
intermediate option like \enquote{kinda right}. This is what study 1
shows. With two alternatives, not many adults judge infelicity with
disjunctive guesses and there are almost no \enquote{wrong} responses.
With three alternatives, \enquote{kinda right} responses pop up. Adult
responses are split between \enquote{kinda right} and \enquote{right}.

\begin{figure}
\centering
\includegraphics{figs/implicaturePlot-1.pdf}
\caption{\label{fig:implicaturePlot}Adult responses to disjunction guesses
like \textit{cat or dog} with 2 and 3 options.}
\end{figure}

If detecting and reacting to underinformativity is the whole story, then
disjunctive guesses should show similar degrees of infelicity,
regardless of how many animals there are on the card. However, the
results of the 3AFC task suggest otherwise. A logistic mixed-effects
model with the random intercepts and slopes for subjects and fixed
effect of disjunction type found that when comparing disjunctive guesses
in the 3AFC task, participants were more likely to choose \enquote{kinda
right} than \enquote{right} when both animals were on the card
(\(\beta\)=-1.22, \(z\)=-2.25, \(p\)=0.02). In other words, participants
judged further infelicity with disjunctive guesses that had both
disjuncts as true. Therefore, it is possible that in some trials when
both disjuncts were true, some participants went through the following
pragmatic reasoning:

\begin{itemize}
\item
  \textbf{Reasoning on Alternatives}: Why did the guesser choose the
  under-informative connective \emph{or} rather than the more
  informative \emph{and}?
\item
  \textbf{Resolution Assumption}: speaker is trying to be as specific as
  possible by resolving the issue of how many animals are on the card.
\item
  \textbf{Exclusivity Implicature}: Given the resolution hypothesis, if
  the speaker had decided that two animals were on the card, they should
  have said \emph{cat and dog}. They did not, so they had decided that
  only one animal is on the card and not both.
\end{itemize}

How does the exclusivity implicature affect participant judgments in the
experimental setting? One possibility is that excluding the correct
response pragmatically is treated like cases of excluding the right
response semantically. For example, guessing \enquote{elephant} when
there is a cat on the card. The prediction is that disjunctive trials
with true disjuncts should receive \enquote{wrong} responses. However,
this prediction was not borne out. Such disjunctive trials are almost
never judged as \enquote{wrong}.

Alternatively, it is possible that adults differentiate incorrect
pragmatics from incorrect semantics (i.e.~falsehood) and punish
incorrect pragmatics less than incorrect semantics. This conclusion is
supported by the response patterns across trial types (figure
\ref{fig:ternaryAdultsPlot}). Trial types that received a
\enquote{wrong} response were those that were false. Pragmatically
infelicitous trial types, namely simple guesses like \emph{cat} or
disjunctive guesses like \emph{cat or dog} when both animals are on the
card, receive \enquote{kinda right} responses. In other words, adults
consider false utterances as \enquote{wrong} guesses but infelicitous
utterances do not reach the level of being \enquote{wrong}; they are
still right even though not completely right. This would explain why the
rates of infelicity (avoiding the \enquote{right} alternative) differ
between 2AFC and 3AFC tasks in disjunctive trials with true disjuncts
(0.18\% vs.~0.62\%).

\subsection{Study 2: Children's 3AFC judgments and open-ended verbal
feedback}\label{study-2-childrens-3afc-judgments-and-open-ended-verbal-feedback}

The goal of this study was to examine children's interpretations of
\emph{and} and \emph{or} in the guessing game and compare them to those
of the adults. Since the 3AFC judgment task in study 1 proved better at
capturing the nuances of adults' pragmatic reasoning, we decided to
first test children using the 3AFC task. We also analyzed children's
open-ended verbal feedback about the guesses in the experimental
context. Both 3AFC judgments and the analysis of children's open-ended
feedback showed that children differentiate existential sentences with
\emph{and} from those with \emph{or}. While the 3AFC task suggested that
children consider disjunctive guesses with true disjuncts as felicitous,
the analysis of their verbal feedback showed otherwise. Children took
issue with such guesses and corrected them, often by mentioning the
stronger alternative \emph{and}. We conclude that the 3AFC task may have
underestimated children's pragmatic competence.

\subsubsection{Methods}\label{methods-1}

\begin{longtable}[]{@{}lllll@{}}
\caption{\label{tab:study2info} Summary of Study 2 Methods}\tabularnewline
\toprule
\begin{minipage}[b]{0.11\columnwidth}\raggedright\strut
Study\strut
\end{minipage} & \begin{minipage}[b]{0.04\columnwidth}\raggedright\strut
N\strut
\end{minipage} & \begin{minipage}[b]{0.21\columnwidth}\raggedright\strut
Age\strut
\end{minipage} & \begin{minipage}[b]{0.17\columnwidth}\raggedright\strut
Mode\strut
\end{minipage} & \begin{minipage}[b]{0.32\columnwidth}\raggedright\strut
Response Option\strut
\end{minipage}\tabularnewline
\midrule
\endfirsthead
\toprule
\begin{minipage}[b]{0.11\columnwidth}\raggedright\strut
Study\strut
\end{minipage} & \begin{minipage}[b]{0.04\columnwidth}\raggedright\strut
N\strut
\end{minipage} & \begin{minipage}[b]{0.21\columnwidth}\raggedright\strut
Age\strut
\end{minipage} & \begin{minipage}[b]{0.17\columnwidth}\raggedright\strut
Mode\strut
\end{minipage} & \begin{minipage}[b]{0.32\columnwidth}\raggedright\strut
Response Option\strut
\end{minipage}\tabularnewline
\midrule
\endhead
\begin{minipage}[t]{0.11\columnwidth}\raggedright\strut
Study 2\strut
\end{minipage} & \begin{minipage}[t]{0.04\columnwidth}\raggedright\strut
42\strut
\end{minipage} & \begin{minipage}[t]{0.21\columnwidth}\raggedright\strut
3;1-5;2 (M = 4;3)\strut
\end{minipage} & \begin{minipage}[t]{0.17\columnwidth}\raggedright\strut
Study Room\strut
\end{minipage} & \begin{minipage}[t]{0.32\columnwidth}\raggedright\strut
Circle (wrong), Little Star (little right), Big Star (right)\strut
\end{minipage}\tabularnewline
\bottomrule
\end{longtable}

\paragraph{Materials and Design}\label{materials-and-design-1}

We used the same set of cards and linguistic stimuli as the ones in
study 1. There were 8 trial types and 2 trials per trial type for a
total of 16 trials. We made two changes to make the experiment more
suitable for children. First, instead of the fictional character Bob, a
puppet named Jazzy played the guessing game with them. Jazzy wore a
sleeping mask over his eyes during the game (Figure \ref{fig:jazzy}).
Second, a pilot study showed that a scale with three alternatives is
better understood and used by children if it is presented in the form of
rewards to the puppet rather than verbal responses such as
\enquote{wrong}, \enquote{a little bit right}, and \enquote{right}, or
even hand gestures such as thumbs up, middle, and down. Therefore, we
placed a set of red circles, small blue stars, and big blue stars in
front of the children. These tokens were used to reward the puppet after
each guess. During the introduction, the experimenter explained that if
the puppet is right, the child should give him a big star, if he is a
little bit right, a little star, and if he is not right, a red circle.

\begin{figure}[!h]

{\centering \includegraphics{figs/jazzy-1} 

}

\caption{The puppet, Jazzy, with and without the sleeping mask.}\label{fig:jazzy}
\end{figure}

\paragraph{Participants and
Procedure}\label{participants-and-procedure-1}

We recruited 42 English speaking children from the Bing Nursery School
at Stanford University. Children were between 3;1 and 5;2 years old
(Mean = 4;3). The experiment was carried out in a quiet room and all
sessions were videotaped. There was a small table and two chairs in the
room. Children sat on one side of the table and the experimenter and the
puppet on the other side facing the child. The groups of circles, small
stars, and big stars were placed in front of the child from left to
right. A deck of six cards was in front of the experimenter. As in study
1 with adults, the children went through three phases: introduction,
instruction, and test.

The goal of the introduction was for the experimenter to show the cards
to the children and make sure they recognized the animals and knew their
names. The experimenter showed the cards to the children and asked them
to label each animal. All children recognized the animals and could
label them correctly. In the instruction phase, children went through
three example trials. The experimenter explained that he was going to
play with the puppet first, so that the child could learn the game. He
removed the six introduction cards and placed a deck of three cards
face-down on the table. From top to bottom (first to last), the cards
had the following images: cat, elephant, cat and dog. He put the
sleeping mask on the puppet's eyes and explained that the puppet is
going to guess what animal is on the cards. He then picked the first
card and asked the puppet: \enquote{\emph{What do you think is on this
card?}} The puppet replied with \enquote{\emph{There is a dog}}. The
experimenter showed the cat-card to the child and explained that when
the puppet is \enquote{not right} he gets a circle. The pilot study had
shown that some children struggle with understanding the word
\enquote{wrong}, so \enquote{not right} was used instead. He then asked
the child to give the puppet a circle. Rewards were collected by the
experimenter and placed under the table to not distract the child. The
second trial followed the same pattern except that the puppet guessed
\enquote{right} and the experimenter invited the child to give the
puppet a big star. In the final trial, the puppet guessed that there is
a cat on the card when the card had a cat and a dog on it. The
experimenter said that the puppet was \enquote{a little right} and asked
the child to give him a little star.

\begin{longtable}[]{@{}lll@{}}
\caption{\label{tab:instruction} Instruction Trials.}\tabularnewline
\toprule
Card & Guess & Reward\tabularnewline
\midrule
\endfirsthead
\toprule
Card & Guess & Reward\tabularnewline
\midrule
\endhead
CAT & There is a cat! & Circle\tabularnewline
ELEPHANT & There is an elephant! & Big Star\tabularnewline
CAT-DOG & There is a dog! & Little Star\tabularnewline
\bottomrule
\end{longtable}

In the test phase, the experimenter removed the three instruction cards
and placed a deck of 16 randomized cards on the table. The experimenter
explained that it was the child's turn to play with the puppet. The test
phase followed the pattern described in the instruction phase.

\paragraph{Offline Annotations}\label{feedbackCoding}

During analysis of the videos, children's linguistic feedback to the
puppet after each guess was categorized into four types: 1. None, 2.
Judgments, 3. Descriptions, and 4. Corrections. The first category
referred to cases where children did not say anything and only rewarded
the puppet. Judgments referred to linguistic feedback such as \emph{you
are right!}, \emph{yes}, \emph{nope}, or \emph{you winned}. Such
feedback only expressed judgments and complemented the rewards.
Descriptions were cases that the child simply mentioned what was on the
card: \emph{cat!}, \emph{dog and elephant!}, \emph{There is a cat and a
dog!} etc. Finally, corrections referred to feedback that provided
\enquote{focused elements} that acted like corrections to what the
puppet had said. Examples include: \emph{Just a cat!}, \emph{Both!},
\emph{The two are!}, \emph{Only cat}, \emph{cat AND dog} (with emphasis
placed on \emph{and}). In trials where the child provided both judgments
as well as descriptions or corrections, we placed the feedback into the
more informative categories, namely description or correction.

\subsubsection{Results}\label{results-1}

\begin{figure}
\centering
\includegraphics{figs/childrenTernaryPlot-1.pdf}
\caption{\label{fig:childrenTernaryPlot}Children's 3AFC judgments in the
connective guessing game.}
\end{figure}

Figure \ref{fig:childrenTernaryPlot} shows the results for children's
3AFC judgments. Starting from the left column, if the mentioned animal
was not on the card (e.g.~elephant), children judged the guess as
\enquote{wrong}. If the animal mentioned (e.g.~cat) was the only animal
on the card, children judged the guess to be \enquote{right}. Here we
ignore the results for trial types in which the animal mentioned was one
of the animals on the card. The reason is that such trials were used in
the instruction phase to introduce the \enquote{little bit right}
guesses, and the results are potentially biased by the instructions.

In conjunctive guesses (e.g. \emph{cat and dog}), when only one of the
animals mentioned was on the card, children judged the guess as
\enquote{wrong} or \enquote{a little bit right}. However, if both
animals were on the card, they judged the conjunctive guess as
\enquote{right}. In disjunctive guesses (e.g. \emph{cat or dog}), when
only one of the animals mentioned was on the card, children considered
the guess \enquote{right} or \enquote{kinda right}. If both animals were
on the card, the disjunctive guess was considered \enquote{right}.

The comparison of conjunction and disjunction trials (last two columns
of figure \ref{fig:childrenTernaryPlot}) shows that overall, children
distinguished between \emph{and} and \emph{or} when one animal was on
the card. Given that the one-animal conjunction trials are false but the
one-animal disjunction trials are true, the difference in response
patterns may suggest that children understood the truth-conditional
differences between \emph{and} and \emph{or}. The truth judgments did
not provide evidence that children differentiated \emph{and} and
\emph{or} when two animals were on the card. Since in the majority of
examples with \emph{or} and two animals, children responded with
\enquote{right}, it is possible to conclude from the 3AFC judgment data
that children did not generate exclusivity inferences in this task.

Figure \ref{fig:childAdultComp} compares the results for children and
adults' 3AFC judgments in the conjunction and disjunction trials. The
major difference between adults and children's responses was disjunctive
trials with two animals on the card. Most children considered such
trials as \enquote{right} while adults considered them as \enquote{kinda
right}. In the next section, we use Bayesian regression modeling to
compare adults' and children's three-alternative responses more
systematically.

\begin{figure}
\centering
\includegraphics{figs/childAdultComp-1.pdf}
\caption{\label{fig:childAdultComp}Comparison of Adults' and Children's 3AFC
judgments.}
\end{figure}

\paragraph{Analysis and Statistical
Modeling}\label{analysis-and-statistical-modeling}

\begin{figure}
\centering
\includegraphics{figs/stanModelPlot-1.pdf}
\caption{\label{fig:stanModelPlot}Coefficients capturing the relevant
comparisons across conditions in 3AFC judgments in Study 1 and 2. In
naming the coefficients like b(OR,2), OR/AND represents the connective
used and the number 1/2 represents the number of animals on the card.
Error bars represent 99\% regions of highest posterior density.}
\end{figure}

We used the R package RStan for Bayesian statistical modeling to fit
separate ordinal mixed-effects logistic models for the children's and
adults' judgments. The response variable had three ordered levels:
\emph{wrong}, \emph{kinda right}, and \emph{right}. The trial types
\emph{One-Animal-OR}, \emph{Two-Animals-OR}, \emph{One-Animal-AND}
constituted the (dummy-coded) fixed effects of the model with
\emph{Two-Animals-AND} set as the intercept. The model also included
by-subject random intercepts. The priors over trial types and the random
intercepts were set to \(\mathcal{N}(0,10)\). We also included
parameters \(C_1\) and \(C_2\), the two cutpoints delimiting the
logistic for 1) \emph{wrong} and \emph{kinda right} and 2) \emph{kinda
right} and \emph{right} responses, drawn with the prior
\(\mathcal{N}(0,1)\).\footnote{We used a tight prior in this case to
  decrease posterior correlations between cutpoints and intercept.} All
four chains converged after 3000 samples (with a burn-in period of 1500
samples).

We made inferences based on the highest-posterior density (HPD)
intervals for the coefficients estimated from each model. Because
predictors are dummy-coded, it's possible to examine contrasts of
interest by computing the difference between coefficients for pairs of
conditions we wish to contrast. In naming the coefficients like b(OR,2),
OR/AND represents the connective used and the number represents the
number of animals on the card. Figure \ref{fig:stanModelPlot} shows the
contrasts of interest: \emph{b(OR, 2)-b(OR,1)} represents the difference
between the estimated coefficients for the disjunction trials with two
animal on the card and those with only one; \emph{b(OR, 2)} represents
the difference between the estimated coefficients for the conjunction
trials with two animals and the disjunction trials with two animals; and
so on.

Overall, adults' and children's estimated coefficients are similar in
sign to one another, though adults' are more extreme. In the conjunction
trials (\emph{b(AND, 2)-b(AND,1)}), children and adults showed a strong
preference for the cards with two animals rather than one. At the same
time, given two animals on the card, children and adults showed a
preference for \emph{and} rather than \emph{or} (\emph{b(OR, 2)}).
However, with only one animal on the card, children and adults preferred
a disjunctive guess (\emph{b(OR, 1)-b(AND,1)}). These results are
compatible with the truth conditions of conjunction and disjunction.

The main difference between adults and children shows up in the contrast
between the disjunctive trial types: two animals vs.~only one
(\emph{b(OR, 2)-b(OR,1)}). On average, children rated disjunction trials
with two animals higher than those with only one. Adults on the other
hand showed the opposite pattern: they rated disjunction trials with two
animals lower. This pattern is compatible with current accounts of
pragmatic development that suggest an absence of implicatures in
children's interpretations. The idea is that while adults strengthen the
disjunctive guess \emph{cat or dog} to \enquote{cat or dog but not
both}, children simply interpret it as \enquote{cat or dog or both}.
Adults are therefore going to rate trials with both disjuncts true
lower.

The slight preference children show for cards with two animals when the
guess is disjunctive is also compatible with the account proposed by
Singh et al. (2016) and Tieu et al. (2016). However, the effect seems
much smaller here than was reported in their studies. The comparison
with conjunction trials makes it clear that overall, children are not
interpreting \emph{or} as having a conjunctive meaning. The effect in
this study can be more accurately described as a preference in judgment
for both disjuncts being true rather than a conjunctive interpretation
of disjunction. The results from children's spontaneous linguistic
feedback make it less likely that children interpretive \emph{or} as a
conjunction. We will discuss this issue further in section
\ref{conjunctive}.

\paragraph{Children's open-ended
feedback}\label{childrens-open-ended-feedback}

\begin{longtable}[]{@{}lll@{}}
\caption{\label{tab:feedbackCat} Definitions and Examples for the Feedback
Categories.}\tabularnewline
\toprule
\begin{minipage}[b]{0.11\columnwidth}\raggedright\strut
Category\strut
\end{minipage} & \begin{minipage}[b]{0.46\columnwidth}\raggedright\strut
Definition\strut
\end{minipage} & \begin{minipage}[b]{0.32\columnwidth}\raggedright\strut
Examples\strut
\end{minipage}\tabularnewline
\midrule
\endfirsthead
\toprule
\begin{minipage}[b]{0.11\columnwidth}\raggedright\strut
Category\strut
\end{minipage} & \begin{minipage}[b]{0.46\columnwidth}\raggedright\strut
Definition\strut
\end{minipage} & \begin{minipage}[b]{0.32\columnwidth}\raggedright\strut
Examples\strut
\end{minipage}\tabularnewline
\midrule
\endhead
\begin{minipage}[t]{0.11\columnwidth}\raggedright\strut
\textbf{None}\strut
\end{minipage} & \begin{minipage}[t]{0.46\columnwidth}\raggedright\strut
no feedback provided to the puppet, only reward\strut
\end{minipage} & \begin{minipage}[t]{0.32\columnwidth}\raggedright\strut
\strut
\end{minipage}\tabularnewline
\begin{minipage}[t]{0.11\columnwidth}\raggedright\strut
\textbf{Judgment}\strut
\end{minipage} & \begin{minipage}[t]{0.46\columnwidth}\raggedright\strut
the child said yes/no, you are right, etc.\strut
\end{minipage} & \begin{minipage}[t]{0.32\columnwidth}\raggedright\strut
\enquote{No!} , \enquote{You are right Jazzy!}\strut
\end{minipage}\tabularnewline
\begin{minipage}[t]{0.11\columnwidth}\raggedright\strut
\textbf{Description}\strut
\end{minipage} & \begin{minipage}[t]{0.46\columnwidth}\raggedright\strut
mentioned the animal(s) on the card\strut
\end{minipage} & \begin{minipage}[t]{0.32\columnwidth}\raggedright\strut
\enquote{elephant}, \enquote{cat and dog}\strut
\end{minipage}\tabularnewline
\begin{minipage}[t]{0.11\columnwidth}\raggedright\strut
\textbf{Correction}\strut
\end{minipage} & \begin{minipage}[t]{0.46\columnwidth}\raggedright\strut
used focus particles like \emph{only}/\emph{just}, emphasized \emph{and}
or used \emph{both}\strut
\end{minipage} & \begin{minipage}[t]{0.32\columnwidth}\raggedright\strut
\enquote{only cat}, \enquote{just elephant}, \enquote{both!},
\enquote{cat AND dog!}\strut
\end{minipage}\tabularnewline
\bottomrule
\end{longtable}

As explained in section \ref{feedbackCoding}, we also categorized and
annotated children's spontaneous and free-form verbal reactions to the
puppet's guesses. Table \ref{tab:feedbackCat} summarizes the definitions
and examples for each category and Figure \ref{fig:feedbackData} shows
the results. We should point out that each trial type had a similar
number of \enquote{None} cases. Some children remained more or less
silent throughout the experiment and only provided rewards to the
puppet. In the next study we ask children to provide feedback explicitly
and therefore we have no \enquote{None} responses. In the discussion and
analysis here we will not comment further on the \enquote{None} category
but focus on the other three categories.

\begin{figure}
\centering
\includegraphics{figs/feedbackData-1.pdf}
\caption{\label{fig:feedbackData}Children's open-ended Feedback. Error bars
represent 95\% confidence intervals.}
\end{figure}

In the leftmost column, when the guessed animal was not on the card
(e.g.~elephant), children either provided judgments like \enquote{No!}
or described what was on the card like \emph{cat} or \emph{cat and dog}.
However, when the guessed animal was the only animal on the card
(e.g.~cat), most children provided a positive judgment like
\enquote{Yes}. When the animal guessed was only one of the animals on
the card, children described what was on the card, for example,
\emph{cat and dog}. Corrections were rare for all these four trial
types.

In the critical trial types with conjunction and disjunction, children
showed a high rate of corrections and description when the guess used
\emph{and} but there was only one animal on the card. In their
corrections, children used the focus particles \emph{just} and
\emph{only} as in \enquote{just a cat} or \enquote{only a cat}. However,
in trial types where conjunction was used and both animals were
depicted, children predominantly provided positive judgments like
\enquote{Yes!} and \enquote{You are right}. Considering disjunctive
guesses like \enquote{cat or dog}, when only one of the animals was on
the card, most children simply described what was on the card, for
example \enquote{cat}. However, when both animals were on the card,
children corrected the puppet by saying \enquote{Both!} or emphasizing
\emph{and} as in \enquote{cat AND dog!}

We performed chi-squared goodness-of-fit tests to compare the feedback
distributions in the critical conditions with \emph{and} and \emph{or}.
Here we focus on those trials (the four bar charts on the right of
Figure \ref{fig:feedbackData}). Children's linguistic feedback showed
three patterns. First, the one-animal conjunctive and two-animal
disjunctive (top left and bottom right) trials contained a higher
proportion of corrections than the other trial types. These were trials
where the guesses were either false or infelicitous. In the conjunction
trials, a comparison of the feedback distribution in one-animal and
two-animal conditions was statistically significant (\(\chi^2\)(3, 83) =
201.65, p \textless{} .0001), suggesting that children gave different
feedback to true and false guesses. A similar numerical trend was
present in the disjunction trials, but it was not significant
(\(\chi^2\)(9, 4) = 12, p = 0.21).

Second, the one-animal disjunctive trials (top right) showed the highest
proportion of \enquote{descriptions}. These are trials in which the
guess is correct but not specific enough: it leaves two possibilities
open. These trials were significantly different from the one-animal
trials for conjunction (\(\chi^2\)(3, 83) = 62.16, p \textless{} .0001).
Finally, the two-animal conjunctive trials (bottom left) showed the
highest proportion of \enquote{judgments} such as \emph{You are right!}.
This was not surprising given that these trials represented the optimal
guessing scenario. These trials had a significantly different feedback
distribution from the matching disjunction trials (\(\chi^2\)(3, 84) =
184.98, p \textless{} .0001).

\subsubsection{Discussion}\label{discussion-1}

In study 2, we used a 3AFC judgment task to test children's
comprehension of logical connectives \emph{and} and \emph{or}. We
compared these results to those found in the 3AFC judgment task of study
1 with adults. The general comparison showed that adults and children
had similar patterns of judgments, except when both disjuncts were true.
In such cases, adults judged the disjunctive guess as not completely
right while most children found it completely right. There was even a
slight preference among children to rewarded the puppet more in such
cases, compared to cases of disjunction when only one disjunct was true.

To consider another measure of children's comprehension, we also looked
at children's spontaneous open-ended verbal feedback to the puppet's
guesses. Our analyses suggested that children recognized false and
infelicitous utterances with the connectives and provided appropriate
corrective feedback. As expected from an adult-like understanding of
connectives, children corrected the puppet most often when there was
only one animal on the card and the guess was conjunctive, or when there
were two animals on the card and the guess was disjunctive. Perhaps the
most important finding was that children increased their corrective
feedback in disjunctive guesses where both disjuncts were true, compared
to those with only one true disjunct. These findings differ from the
results of the 3AFC judgment task which suggested that children did not
find any infelicity with disjunctive guesses when both disjuncts were
true.

The analysis of children's open-ended feedback raises two important
issues. First, as we mentioned before, it runs counter to what the 3AFC
judgment task suggests with respect to exclusivity implicatures. The
forced-choice task suggests that children find such underinformative
utterances as unproblematic while analysis of their spontaneous feedback
shows that they provided more corrections to such utterances. Second, a
common explanation for why children fail to derive implicatures is that
they cannot access the stronger alternative to the disjunction
\emph{or}, namely \emph{and} (Barner, Brooks, \& Bale, 2011). However,
in the context of the guessing game, some children explicitly mentioned
the word \emph{and}, as the word the puppet should have said instead of
\emph{or}. Interestingly, these children continued to reward the puppet
and considered the guess \enquote{right}. This raises the possibility
that children's forced-choice truth value judgments, whether with two or
three alternatives, do not fully reflect their pragmatic knowledge. In
study 3, we used both a 2AFC truth judgment task and an analysis of
children's open-ended feedback. If the findings of study 2 were on the
right track, we expected to replicate the same pattern in study 3,
namely that the analysis of children's open-ended feedback should
provide more evidence that children are sensitive to pragmatic
violations than the results of the 2AFC judgments.

\subsection{Study 3: Children's 2AFC judgments and open-ended
feedback}\label{study-3-childrens-2afc-judgments-and-open-ended-feedback}

This study used the same paradigm as study 2 but focused on children's
open-ended feedback and aimed at replicating the findings in study 2.
The main hypothesis was that four-year-olds provide corrective feedback
to the puppet if both disjuncts are true, but they do not consider this
infelicity to be grave enough to render the guess itself
\enquote{wrong}. The main hypothesis along with relevant analyses and
predictions were preregistered in an \enquote{As Predicted}
format\footnote{The As Predicted pdf document is accessible at
  \url{https://aspredicted.org/x9ez2.pdf}.}. The study used a 2AFC
judgment task to compare with the open-ended feedback results. The
prediction was that children would provide corrective feedback to the
puppet when both disjuncts were true, yet consider the guess
\enquote{right} and not reflect this infelicity in their truth value
judgments. This is what the study found.

\subsubsection{Methods}\label{methods-2}

\begin{longtable}[]{@{}lllll@{}}
\caption{\label{tab:study3info} Summary of Study 1, 2, and 3
Methods}\tabularnewline
\toprule
\begin{minipage}[b]{0.19\columnwidth}\raggedright\strut
Study\strut
\end{minipage} & \begin{minipage}[b]{0.02\columnwidth}\raggedright\strut
N\strut
\end{minipage} & \begin{minipage}[b]{0.20\columnwidth}\raggedright\strut
Age\strut
\end{minipage} & \begin{minipage}[b]{0.11\columnwidth}\raggedright\strut
Mode\strut
\end{minipage} & \begin{minipage}[b]{0.32\columnwidth}\raggedright\strut
Response Options\strut
\end{minipage}\tabularnewline
\midrule
\endfirsthead
\toprule
\begin{minipage}[b]{0.19\columnwidth}\raggedright\strut
Study\strut
\end{minipage} & \begin{minipage}[b]{0.02\columnwidth}\raggedright\strut
N\strut
\end{minipage} & \begin{minipage}[b]{0.20\columnwidth}\raggedright\strut
Age\strut
\end{minipage} & \begin{minipage}[b]{0.11\columnwidth}\raggedright\strut
Mode\strut
\end{minipage} & \begin{minipage}[b]{0.32\columnwidth}\raggedright\strut
Response Options\strut
\end{minipage}\tabularnewline
\midrule
\endhead
\begin{minipage}[t]{0.19\columnwidth}\raggedright\strut
Study 1 - Part 1\strut
\end{minipage} & \begin{minipage}[t]{0.02\columnwidth}\raggedright\strut
57\strut
\end{minipage} & \begin{minipage}[t]{0.20\columnwidth}\raggedright\strut
Adults\strut
\end{minipage} & \begin{minipage}[t]{0.11\columnwidth}\raggedright\strut
Online (Mturk)\strut
\end{minipage} & \begin{minipage}[t]{0.32\columnwidth}\raggedright\strut
Wrong, Right\strut
\end{minipage}\tabularnewline
\begin{minipage}[t]{0.19\columnwidth}\raggedright\strut
Study 1 - Part 2\strut
\end{minipage} & \begin{minipage}[t]{0.02\columnwidth}\raggedright\strut
52\strut
\end{minipage} & \begin{minipage}[t]{0.20\columnwidth}\raggedright\strut
Adults\strut
\end{minipage} & \begin{minipage}[t]{0.11\columnwidth}\raggedright\strut
Online (Mturk)\strut
\end{minipage} & \begin{minipage}[t]{0.32\columnwidth}\raggedright\strut
Wrong, Kinda Right, Right\strut
\end{minipage}\tabularnewline
\begin{minipage}[t]{0.19\columnwidth}\raggedright\strut
Study 2\strut
\end{minipage} & \begin{minipage}[t]{0.02\columnwidth}\raggedright\strut
42\strut
\end{minipage} & \begin{minipage}[t]{0.20\columnwidth}\raggedright\strut
3;1-5;2 (M = 4;3)\strut
\end{minipage} & \begin{minipage}[t]{0.11\columnwidth}\raggedright\strut
Study Room\strut
\end{minipage} & \begin{minipage}[t]{0.32\columnwidth}\raggedright\strut
Circle (Wrong), Little Star (Little Right), Big Star (Right)\strut
\end{minipage}\tabularnewline
\begin{minipage}[t]{0.19\columnwidth}\raggedright\strut
Study 3\strut
\end{minipage} & \begin{minipage}[t]{0.02\columnwidth}\raggedright\strut
50\strut
\end{minipage} & \begin{minipage}[t]{0.20\columnwidth}\raggedright\strut
3;6-5;9 (M = 4;7)\strut
\end{minipage} & \begin{minipage}[t]{0.11\columnwidth}\raggedright\strut
Study Room\strut
\end{minipage} & \begin{minipage}[t]{0.32\columnwidth}\raggedright\strut
Yes (Right)/No (Wrong) - Open-ended Feedback\strut
\end{minipage}\tabularnewline
\bottomrule
\end{longtable}

\paragraph{Materials and Design}\label{materials-and-design-2}

Study 3 was similar to Study 2 but differed in how children provided
their judgments. Based on the findings in Study 2, we focused on verbal
judgments and feedback, instead of rewards. We used two different ways
of measuring children's judgments. First, we encouraged children to
provide verbal feedback to the puppet. They were asked to say
\enquote{yes} when the puppet was right, and \enquote{no} when he was
not. They were also encouraged to help the puppet say it better when he
was not right. After children were done with this initial open-ended
feedback, for each trial we asked a forced choice yes/no judgment
question: \enquote{Was Jazzy (the puppet) right?}. This question
elicited a \enquote{yes} or \enquote{no} response for each trial
independent of their earlier open-ended response. These two measures
allowed me to compare open-ended and forced-choice judgments.

\paragraph{Participants and
Procedure}\label{participants-and-procedure-2}

We recruited 50 English speaking children from the Bing Nursery School
at Stanford University. Children were between 3;6 and 5;9 years old
(Mean = 4;7). The setup and procedure were similar to Study 2, except
there were no rewards on the table. As before, participants sat through
three phases: introduction, instruction, and test. The introduction
phase made sure children knew the names of the animals on the cards. In
the instruction phase, they received four training trials, as shown in
Table \ref{tab:instructionStudy3}.

As in Study 2, the experimenter put a sleeping mask over the puppet's
eyes and explained that Jazzy (the puppet) was going to guess what
animal was on the cards. He then picked the first card and asked the
puppet: \enquote{\emph{What do you think is on this card?}} The puppet
replied with \enquote{\emph{There is a dog}}. The experimenter showed
the cat-card to the child and said: when Jazzy is \emph{not right}, tell
him \enquote{no}. He then asked the child to say \enquote{no} to the
puppet. The second trial followed the same pattern except that the
puppet guessed \emph{right} and the experimenter invited the child to
say \enquote{yes} to the puppet. There were two more instruction trials
before the test phase began. This contained 16 randomized trials, half
of which contained guesses with the words \emph{and} and \emph{or}. The
randomization code as well as the details of the methods are on the
online repository for this dissertation at
\href{https://github.com/jasbi/jasbi_dissertation_LearningDisjunction/blob/master/connective_comprehension/study3/0_methods/children}{https://github.com/jasbi/jasbi\_dissertation\_LearningDisjunction}.

\begin{longtable}[]{@{}lll@{}}
\caption{\label{tab:instructionStudy3} Instruction Trials for Study
3.}\tabularnewline
\toprule
Card & Guess & Response\tabularnewline
\midrule
\endfirsthead
\toprule
Card & Guess & Response\tabularnewline
\midrule
\endhead
CAT & there is a dog! & No!\tabularnewline
ELEPHANT & there is an elephant! & Yes!\tabularnewline
DOG-ELEPHANT & there is a cat! & No!\tabularnewline
DOG & there is a dog! & Yes!\tabularnewline
\bottomrule
\end{longtable}

\subsubsection{Results}\label{results-2}

We first look at the results of the 2AFC judgement task for each trial
type and compare them to those of the adults' in Study 1. Then we
analyze children's open-ended responses and compare them to the forced
choice responses obtained in the same trial types. For the 2AFC
judgments we excluded 26 trials (out of total 800) where children either
did not provide a Yes/No response or provided both (i.e. \enquote{Yes
and No}). The exclusions were almost equally distributed among different
types of guesses and cards. In the analysis of children's open-ended
feedback, we excluded 8 trials (out of total 800) where children either
did not provide any feedback or their feedback could not be categorized
into the existing categories.

\paragraph{Two-Alternative Forced Choice
Judgments}\label{two-alternative-forced-choice-judgments}

\begin{figure}
\centering
\includegraphics{figs/Study3tvjtPlot-1.pdf}
\caption{\label{fig:Study3tvjtPlot}Children's binary truth value judgments.}
\end{figure}

Figure \ref{fig:Study3tvjtPlot} shows children's 2AFC judgments. In the
leftmost column, when the animal guessed was not on the card
(e.g.~elephant), children considered the guess \enquote{wrong}. When the
animal guessed was the only animal on the card (e.g.~cat), children
considered the guess \enquote{right}. However, if the animal guessed
(e.g.~cat) was only one of the animals on the card, children were
equally split between \enquote{wrong} and \enquote{right} judgments. On
the other hand, almost all adults considered such guesses
\enquote{right} in their 2AFC judgments (Figure
\ref{fig:binaryAdultsPlot}). In such trial types, children seem to
interpret the guess \enquote{there is a cat} as \enquote{there is
\textbf{only} a cat}, while adults do not. This difference between
children and adults is unexpected for a theory of meaning acquisition
that assumes children are overall more logical or literal as
interpreters than adults (Noveck, 2001).

In the trials with \emph{and} and \emph{or}, children's judgments were
similar to those of adults. Figure \ref{fig:BinaryPlotComp} compares
adults' and children's 2AFC judgments. In trials with conjunction, when
only one of the animals was on the card, most children considered the
guess \enquote{wrong}. This is similar to adults' judgments, but
different in extent: adults were more consistent and unanimous in
rejecting such guesses. A mixed effects logistic regression with the
fixed effect of age category (adult vs.~child) and random effect of
subject found no significant difference between adults' and children's
responses in such trials (see Table \ref{tab:statsStudy3}, Conjunction -
One Animal).

\begin{longtable}[]{@{}lcccc@{}}
\caption{\label{tab:statsStudy3} Mixed effects logistic models for
conjunction and disjunction trials when only one disjunct was true, in
2AFC judgments of adults and children, using \texttt{glmer} in R's lme4
package. Formula:
\(Response \sim Age Category + (1|Subject)\).}\tabularnewline
\toprule
Trial Data & Coefficient & Standard Error & Z-Value &
P-value\tabularnewline
\midrule
\endfirsthead
\toprule
Trial Data & Coefficient & Standard Error & Z-Value &
P-value\tabularnewline
\midrule
\endhead
Conjunction - One Animal & -2.05 & 2.86 & -0.72 & 0.47\tabularnewline
Disjunction - One Animal & 1.34 & 1.79 & 0.75 & 0.45\tabularnewline
\bottomrule
\end{longtable}

In conjunctive guesses where both animals were on the card, both
children and adults were unanimous in considering the guess
\enquote{right}. In disjunctive trials when only one of the animals was
on the card, most children considered the guess \enquote{right}. This is
again similar to adults but differs from them in extent: adults more
consistently and unanimously judged such guesses as \enquote{right}. Yet
again, a mixed effects logistic regression with the fixed effect of age
(adult vs.~child) and random effect of subject found no significant
difference between adults' and children's responses in such trials (see
Table \ref{tab:statsStudy3}, Disjunction - One Animal). Adults and
children showed almost identical patterns of judgments in trials where
there was two animals on the card and the guess used the connective
\emph{or}. Children and adults did not differ in their rate of rejecting
disjunctive guesses when both disjuncts were true.

\begin{figure}
\centering
\includegraphics{figs/BinaryPlotComp-1.pdf}
\caption{\label{fig:BinaryPlotComp}The comparison of the 2AFC judgment task
for conjunction and disjunction trials in adults (study 1) and children
(study 3).}
\end{figure}

Finally, there is a small but significant preference in children's
judgments of disjunctive statements for both disjuncts to be true.
Comparing the disjunctive trials with one animal and two animals on the
card, a mixed-effects logistic model with the fixed effect of
disjunction type and the random effect of subjects found that children
had a slight preference for both animals to be on the card (\(b\)= 1.85,
\(se\)= 0.56, \(z\)= 3.32, \(p < 0.001\) ). There was a similar small
trend in children's three-alternative judgments in study 2. While this
was quite small compared to the other effects observed in these studies,
it nevertheless indicated a difference between children's and adults'
judgments. We return to this in more detail in section \ref{conjunctive}
of the General Discussion.

\paragraph{Open-ended Feedback}\label{open-ended-feedback}

\begin{figure}
\centering
\includegraphics{figs/feedbackStudy3-1.pdf}
\caption{\label{fig:feedbackStudy3}Children's Open-ended Feedback in Study
3. Error bars represent 95\% confidence intervals.}
\end{figure}

Figure \ref{fig:feedbackStudy3} shows the distribution of children's
feedback to the puppet in Study 3 (see Table \ref{tab:feedbackCat} for
the definitions and examples of feedback categories). There were no
\enquote{None} responses in this study since the experimenter explicitly
asked children to provide feedback to the puppet. The distribution of
the responses in the other three categories (Judgment, Description, and
Correction) revealed a successful replication of Study 2.

Children's feedback showed four main patterns. First when the puppet
guessed an animal not on the card (e.g. \emph{There is an elephant!}),
there is a split pattern between negative judgments like \emph{No!} and
simply mentioning the animal on the card, e.g. \emph{Cat!}. Children
provided no corrections on such trials, at leat the way we have defined
them. Second, almost all children responded with positive judgments like
\emph{Yes!} when the puppet's guess accurately matched what was on the
card. This was the case in trials where there was only one animal on the
card (e.g.~cat) and the puppet mentioned it (e.g. \emph{There is a
cat!}), as well as trials where there were two animals on the card and
the puppet mentioned both with a conjunction (e.g. \emph{There is a cat
and a dog!}). Third, children provided the largest number of corrective
feedback in trials where the guess was either false or infelicitous.
These included three trial types: (a) the ones where there were two
animals on the card (e.g.~cat and dog) but the puppet only guessed one
(e.g. \emph{There is a cat!}); (b) the ones where the puppet guessed two
animals with conjunction (e.g. \emph{There is a cat and a dog!}) but
only one of them was on the card (e.g.~cat); and (c) the ones where
there were two animals on the card (e.g.~cat and dog), and the puppet
guessed both but used a disjunction (e.g. \emph{There is a cat or a
dog!}). Finally, there was a pattern of feedback unique to disjunctive
trials (e.g. \emph{There is a cat or a dog!}) with only one animal on
the card (e.g.~cat). In such cases, almost all children simply named the
animal on the card (e.g. \emph{Cat!}).

\begin{figure}
\centering
\includegraphics{figs/study3JudgmentPlot-1.pdf}
\caption{\label{fig:study3JudgmentPlot}Children's open-ended feedback to the
puppet's guesses. The x-axis shows whether children spontaneously
provided a yes (green), no (red), or other response (grey).}
\end{figure}

Figure \ref{fig:study3JudgmentPlot} breaks down children's open-ended
feedback based on whether children said \emph{Yes!}, \emph{No!}, or said
something else. Responses that were not yes/no judgments are grouped in
a middle category shown with a dash. The goal here is to compare
children's open-ended judgments with their forced choice judgments shown
in Figure \ref{fig:Study3tvjtPlot}. Children's open-ended judgments and
their forced choice judgments in study 3 show similar patterns for all
types of guesses except for disjunctive ones. In trials that the puppet
guessed with \emph{or}, the vast majority of children refused to provide
a yes/no judgment when they were not forced to. Instead, they described
the animal on the card or provided corrections to the puppet's
infelicitous disjunctive guess.

One way to interpret these results is that disjunctive guesses (with at
least one disjunct true) are considered neither right nor wrong by
almost all children. When children were forced to provide wrong/right
responses in the experimental context, some conformed to the adult
patterns of judgment and some did not. However, it is possible that such
deviations from adult judgments do not reflect differences in the
comprehension of disjunction, but rather differences in how children map
their comprehension of disjunction onto the notions of \enquote{right}
and \enquote{wrong} in a forced choice judgment task. In other words, it
is possible that children and adults only differ in how they behave when
they are forced to respond with a fixed set of options.

\begin{figure}
\centering
\includegraphics{figs/correctivePlot-1.pdf}
\caption{\label{fig:correctivePlot}Children's feedback categories in
disjunction trials.}
\end{figure}

Figure \ref{fig:correctivePlot} shows the proportion of feedback
categories other than yes/no judgments on the x-axis. My goal here is to
display the trial types with corrective feedback (blue and red). These
trial types include: (1) conjunction when only one conjunct is true
(e.g.~guess: \emph{There is a cat and a dog!}, card: cat), (2)
disjunction when both disjuncts are true (e.g.~guess: \emph{There is a
cat or a dog}, card: cat and dog), and (3) simple guesses when two
animals were on the card (e.g. \emph{There is a cat!}, card: cat and
dog). These trial types involved guesses that were either false or
infelicitous. Furthermore, the type of corrective feedback children
provided matched the type of mistakes made in the guesses. With
conjunctive guesses (e.g.~There is a cat and a dog!``) when there was
only one animal on the card (e.g.~cat), children provided exclusive
corrections (e.g. \emph{Just/only a cat!}), suggesting that the other
animal (e.g.~dog) should have been excluded. When two animals were on
the card (e.g.~cat and dog) and the puppet used a disjunctive guess
(e.g. \emph{There is a cat or a dog!}), or simple guess (e.g. \emph{Ther
is a cat!}), children provided inclusive feedback, suggesting that
another animal should have been included. This is particularly notable
in the case of disjunction since both animals were mentioned, but
children still emphasized that the connective \emph{and} should have
been used, or that both animals mentioned were on the card.

\subsubsection{Discussion}\label{discussion-2}

Study 3 measured children's comprehension of logical connectives in two
ways: First, with analyzing their open-ended feedback and second, with a
two-alternative forced choice task. First, we asked children to say
\emph{yes} to the puppet if he was right and \emph{no} if he was wrong.
However, children could provide any form of feedback they wanted.
Second, we followed children's open-ended feedback with a
two-alternative forced choice question: \emph{Was the puppet right?}
This way, we could measure children's comprehension in two different
ways in the same trial. Ideally, both measures should show similar
results. However, the findings were similar for conjunctive guesses, but
not disjunctive ones. Children avoided binary right/wrong feedback with
disjunction and preferred to provide more nuanced feedback.

The 2AFC responses followed the predicted pattern: conjunctive guesses
were judged wrong if only one conjunct was true, and right if both were
true. Disjunctive guesses were judged right whether one or both
disjuncts were true. There was no significant difference in the 2AFC
task between the responses of children and those of adults in Study 1.

Children's open-ended feedback in Study 3 replicated the findings of
Study 2. Children provided more corrective feedback in false and
infelicitous trials than in true and felicitous ones. The corrective
feedback was tailored to the puppet's mistake. If the puppet used a
conjunction when there was only one animal on the card, children pointed
out that the other animal should have been excluded from the guess. They
used the exclusive adverbials \emph{just} and \emph{only} in their
feedback. If the puppet used a disjunction when both animals were on the
card, children stressed \emph{and} or \emph{both}, implying that both
animals should have been included.

While the 2AFC results suggested that children took no issue with
disjunctive guesses when both disjuncts are true, the analysis of their
corrective feedback showed that they provide appropriate corrections in
such cases and emphasize that the connective \emph{and} would have been
a better guess. Taking both measures together, we conclude that even
though children are aware of the problem with such guesses, they do not
consider them \emph{wrong}. These results are similar to those we
reported for adults in Study 1.

\subsection{General Discussion}\label{general-discussion}

We reported three studies on adults and four-year-olds' comprehension of
the logical connectives \emph{and} and \emph{or}. The first study used
two- and three-alternative forced choice judgment tasks with adults. In
the 2AFC task, adult interpretations closely matched the semantic
accounts of \emph{and} and \emph{or} as conjunction and inclusive
disjunction. The 2AFC judgments did not register robust signs of
pragmatic infelicities. However, the 3AFC judgments showed signs of
pragmatic infelicities, especially in disjunctive guesses with true
disjuncts. When two animals where on the card (e.g.~cat and dog) and the
guess used \emph{or} (e.g. \emph{There is a cat or a dog!}),
participants were more likely to choose \enquote{kinda right} rather
than \enquote{right}.

The second study used a 3AFC judgment task with four-year-old children.
It also included an exploratory analysis of children's open-ended verbal
feedback to the puppet in the experimental setting. Children's
interpretations were similar to those of adults in the 3AFC task and
only differed for pragmatically infelicitous disjunctions. When both
disjuncts were true, adults tended to judge disjunctive guesses as
\enquote{kinda right}. This was evidence for the pragmatic infelicity of
such guesses. While, children judged such disjunctive statement as
\enquote{right}, the analysis of their open-ended feedback showed that
they took issue with such statements as well, and provided appropriate
corrective feedback.

In the third study, we focused on eliciting open-ended verbal feedback
from children and followed it with a 2AFC task. In the 2AFC task,
children's responses reflected the semantics of connectives as
conjunction and inclusive disjunction. There was no significant
difference between children and adults in the two-alternative judgments.
Since the 2AFC task appeared to be a good indicator of semantic
knowledge, it seemed reasonable to conclude that adults and
four-year-olds displayed similar semantic knowledge of the connectives.
Analysis of the children's open-ended feedback replicated the findings
in study 2. Children provided more corrective feedback in false and
pragmatically infelicitous trials with logical connectives than in
felicitous trials. The comparison of the 2AFC task and children's
open-ended responses showed that children are sensitive to the
infelicity of disjunctions with true disjuncts, even though they
consider them to be \enquote{right} guesses.

Overall, we did not find any major differences between adults' and
four-year-old children's interpretations of logical connectives
\emph{and} and \emph{or} in the context of the guessing game. However,
there were two minor differences. First, we found that in both 2AFC and
3AFC judgment tasks, children showed a small preference for disjunctions
with both disjuncts true rather than only one. Adults on the other hand
showed the opposite pattern: they preferred disjuncts with only one
disjunct true. Second, in both 2AFC and 3AFC judgment tasks, children
rated disjunctions with both disjuncts true higher than adults did. That
is, they considered utterances like \emph{There is a cat or a dog} when
both animals were on the card \enquote{right} more often than adults
did. Here we will discuss these two differences and their possible
causes in more detail.

\subsubsection{Preference for True Disjuncts}\label{conjunctive}

First for some children, there was a small preference for both disjuncts
being true, compared to only one. This effect is similar in kind but not
magnitude, to an effect that Singh et al. (2016) and Tieu et al. (2016)
reported. In our study this effect is quite small while Singh et al.
(2016) and Tieu et al. (2016) seem to have found bigger effects. Based
on this, Singh et al. (2016) proposed that many children at this
age-range have a pragmatically driven conjunctive interpretation of
disjunction. In short, due to a non-adult like alternative set to the
connective \emph{or}, children strengthen a disjunctive statement
pragmatically and derive a conjunction. The studies reported here
provide no support for this proposal. In both 2AFC and 3AFC judgments,
children clearly differentiated between disjunctive and conjunctive
guesses. Furthermore, analysis of children's open-ended feedback showed
distinctly different response patterns for conjunction and disjunction.
More importantly, the open-ended feedback to disjunctive guesses showed
the opposite pattern to that predicted by the conjunctive hypothesis.
Children took issue with disjunctions that had both disjuncts true and
provided more corrective feedback in such cases. Therefore, the findings
from Singh et al. (2016) and Tieu et al. (2016) may be a product of
experimental design rather than a real reflection of children's
comprehension of the connectives.

However, even if this small preference for true disjuncts is not due to
the method of measurement, it can be accounted for in several other ways
that have not yet been successfully ruled out. First, the conjunctive
interpretation may not be due to a faulty pragmatic computation, but
rather a default conjunctive interpretation when the connective is not
properly heard, understood or is unknown. To check this hypothesis, it
should be possible to test children's comprehension of novel or noisy
connectives. A novel coordination like \emph{cat dax dog} with
\emph{dax} as a nonce connective could well be interpreted as a
conjunction. Such a result would suggest that in studies with high
cognitive demand, children may default to a conjunctive interpretation
if they miss the relevant connective. Second, the conjunctive preference
could be due to some children's preference for the linguistic labels to
match the animals on the card (or more generally a match between
linguistic description and the state of the world). This hypothesis is
consistent with the results in the other trial type that had a mismatch
in the number of animals and the guess, where the guess was still
technically true: simple guesses (e.g.~there is a cat) with two animals
(e.g.~cat and dog). Children were equally split between \enquote{wrong}
and \enquote{right} in their judgments here, while adults considered
such guesses \enquote{right}. In light of these alternative
explanations, we are hesitant to attribute this small preference to a
pragmatically driven conjunctive interpretation of disjunction.

\subsubsection{Lack of infelicity with true disjuncts in the forced
choice
tasks}\label{lack-of-infelicity-with-true-disjuncts-in-the-forced-choice-tasks}

The second difference between adults and children emerged in the 3AFC
judgment task: in disjunctive trials (e.g. \emph{There is a cat or a
dog}) with two animals (e.g.~cat and dog), adults were more likely to
choose \enquote{kinda right} than children were. Children mostly chose
\enquote{right}. This response pattern has been taken to mean that
children found no infelicity with such disjunctions or that they did not
\enquote{derive an exclusivity implicature}. The absence of an
infelicity/implicature is consistent with the generalization that
children are more likely than adults to interpret scalar terms
literally, and that children do not compute implicatures or judge
infelicity to the same \textbf{rate} that adults do (Pouscoulous \&
Noveck, 2009, Katsos (2014)). But why is that?

There have been at least three major proposals to account for children's
low rate of implicatures: 1. processing difficulty (Pouscoulous, Noveck,
Politzer, \& Bastide, 2007; Reinhart, 2004) 2. non-adult-like lexical
entry (Barner et al., 2011; Horowitz, Schneider, \& Frank, 2017) and 3.
pragmatic tolerance (Katsos \& Bishop, 2011). Here we show that none of
these accounts can provide a satisfactory explanation of the results in
this study.

\textbf{1. Processing difficulty.} First, processing accounts locate the
problem in children's processing capacities such as working memory. They
suggest that pragmatic computations are cognitively taxing and children
lack the appropriate processing resources to carry them out
appropriately. A prediction of processing accounts (at least in their
current format) is that children will show reduced implicature
computations for all types of implicatures -- scalar or ad-hoc. This
prediction was not borne out in our experimental results here. In Study
3, children were much more likely than adults to call a simple guess
(e.g. \emph{There is a cat!}) \enquote{wrong} if there were two animals
on the card (e.g.~cat and dog). Processing accounts do not predict that
children may derive implicatures at a higher rate than adults but this
is what we found, at least for the traditional interpretation of the
judgment task.

\textbf{2. Non-adult-like Lexicon.} Several proposals blame the
structure of the child's lexicon for the alleged failure in deriving
implicatures. The assumption is that the child's lexical entry for
scalar items must include three elements for successful derivation: 1.
the semantics of the weak term (e.g. \emph{some}, \emph{or}) 2. the
semantics of the strong term (e.g. \emph{all},\emph{and}); and possibly
3. a scale that recognizes the stronger term as an alternative to the
weaker one (e.g. \textless{}\emph{some}, \emph{all}\textgreater{},
\textless{}\emph{or}, \emph{and}\textgreater{}). Each of these elements
have been pinpointed as the source of the problem in previous studies
(Barner et al., 2011; Horowitz et al., 2017; Katsos \& Bishop, 2011).
However none of them seem to apply to the results reported here.

If children in this study lack the semantics of the connective
\emph{or}, we would expect them to either perform at chance or default
to a conjunctive interpretation. Neither prediction was borne out in
studies 2 and 3. Furthermore, children's free-form linguistic feedback
in both studies suggested that children understood disjunction well
enough to provide relevant feedback. So this explanation seems unlikely.
The problem cannot be that children do not know the meaning of
\emph{and} either. Children's performance in both study 2 and 3 for
conjunction trials show that they understand its meaning very well.
Finally, while it is possible that children lacked the appropriate
lexical scale and could not access the stronger alternative, this
explanation cannot be the whole story. Several children in both studies
stressed the word \emph{and} in their verbal feedback, suggesting that
the puppet should have used the stronger term instead. However, they
still judged the puppet's guess as \enquote{right}. If children could
not access the stronger term, they could not mentioned it in their
feedback either.

\textbf{3. Pragmatic Tolerance.} Katsos \& Bishop (2011) suggested that
children tend to tolerate pragmatic infelicities more than adults. They
showed that when children were provided with a 2AFC judgment task, they
considered a description with the scalar term \emph{some} as
\enquote{right} when \emph{all} was more informative (e.g. \emph{The
turtle played with some of the balls.}, Scene: the turtle played with
all the balls.) However, when they are presented with three options
(small, big, and huge strawberries) in a 3AFC task, they choose the
middle option in the same type of trials. They argue that children
tolerate pragmatic infelicities and do not regard them as
\enquote{wrong}. As in a processing account, the tolerance account
predicts that scalar and ad-hoc implicatures will be similarly affected.
However, our results did not match those of Katsos \& Bishop (2011).
When children were presented with a 3AFC task, they chose the highest
reward (and not the middle option) for uses of \emph{or} when \emph{and}
was more informative. Second, and more importantly, we found different
patterns for exhaustive and scalar inferences as mentioned before. This
is not predicted by the tolerance account unless we assume that children
are more tolerant towards violations of scalar inferences than they are
towards exhaustive ones. While this is not currently assumed in the
literature, it is a possible adjustment. However, we would like address
this issue by focusing on another related factor: the role of
measurement in estimates of children's pragmatic capacity (Katsos,
2014). Several observations in the current studies provide support for
the hypothesis that methodological issues, and more specifically issues
of measurement contribute to the differences found between adults and
children in pragmatic capacity. First, Study 1 showed that even for
adults, the estimates of adult infelicity rates may differ based on the
number of alternatives in the forced choice task. A 2AFC task
underestimated adults' sensitivity to pragmatic infelicity. In fact, in
a follow up study, we systematically varied the number of response
options and replicated the results presented here (see Jasbi, Waldon,
and Degen in press). Second, children's open-ended linguistic feedback
in the experimental context better reflected their sensitivity to
pragmatic nuances than the forced-choice judgment tasks. Third, children
showed a higher rate of infelicity judgments for cases of ad-hoc
implicatures (simple guesses with two animals on the card) than adults
did. While a difference in sensitivity to ad-hoc vs.~scalar implicatures
has been reported and argued for before (Horowitz et al., 2017; Stiller,
Goodman, \& Frank, 2015), a higher sensitivity than adults is not
predicted by any of the current accounts.

In order to better understand the differences between adults and
children's pragmatic capacities, it is necessary to have a good
understanding of how our measurements affect estimates of adults and
children's performance in the experimental tasks. Children may be no
more capable of making exhaustive inferences than adults and no less
capable of making scalar inferences either. They may simply have a
different construal of the wrong-right scale and of what the
forced-choice task is about. The concepts \enquote{right} and
\enquote{wrong} are as much subject to developmental change and
differences between adults and children as are scalar items that
constitute the focus of our studies. It is possible that children's
understanding of what constitutes as \enquote{right} or \enquote{wrong}
does not fully conform to that of adults. However, it remains to be
established what these differences are and how they affect the estimates
of children's pragmatic abilities. It is important to point out that
such issues of measurement could be the culprit behind both children's
seemingly slight preference for true disjuncts described earlier and the
lack of infelicity judgments when both disjuncts are true.

\textbf{A General Approach for Measuring Implicature/Infelicity Rate}

Methodological issues are nothing new in developmental studies and
language acquisition. Creating better measures of children's linguistic
capacities has always been a major concern for researchers in the field.
Our goal here is to propose some future steps that can address possible
methodological issues in assesssing children's pragamtic competence.

As Pouscoulous \& Noveck (2009) and Katsos (2014) have suggested, the
central issue is \enquote{the rate} at which children and adults
manifest pragmatic reasoning in the experimental setting. No one doubts
children's capacity to perform such computations. At issue is the extent
to which children and adults compute specific implicatures. The claim is
that children perform such computations less often than adults; or that
children do not perform such computations where adults normally do. In
the previous section, we discussed some factors that might account for
these differences including processing demands, the structure of the
lexicon, tolerance, as well as issues of measuring adults and children's
comprehension. As Katsos (2014) pointed out, it seems reasonable to
assume that all these factors play some part here. What matters is the
degree to which each contributes to the outcome.

\begin{figure}
\centering
\includegraphics{figs/implicatureGraph-1.pdf}
\caption{\label{fig:implicatureGraph}Factors that could affect pragmatic
computations and the estimates of these computations in the experimental
settings}
\end{figure}

Figure \ref{fig:implicatureGraph} shows the factors that affect
pragmatic computations as well as the observations of the rate of
pragmatic computations in an experiment. First it is important to
distinguish between factors that affect pragmatic computations and those
that affect the observed rate in an experimental setting. As we showed
in Study 1, given the number of alternatives in the forced choice task
(2AFC vs.~3AFC), we may get different estimates of adults' rate of
infelicity judgments, but we cannot to assume that there is a difference
in adults' pragmatic capacities in these two tasks. A similar situation
exists when we compare children's forced choice measures of infelicity
and their open-ended feedback. In disjunctive trials where both
disjuncts are true, the forced choice tasks show no sign of children
detecting infelicity while the open ended responses show that children
are sensitive to the infelicity of disjunction when a conjunction would
have been more appropriate.

\subsection{Conclusion}\label{conclusion}

To conclude, the studies presented here did not provide evidence for a
substantial difference between adults and three-to-five-year-old
children in their \textbf{semantic} knowledge of the logical connectives
\emph{and} and \emph{or}. The results were highly consistent with the
current accounts that posit the semantics of \emph{and} as conjunction
and \emph{or} as inclusive disjunction. With respect to pragmatic
knowledge, the three-alternative forced choice judgment task showed that
adults are sensitive to the infelicity of disjunctive statements when
both disjuncts are true. We also showed that the three-alternative
judgment task failed to register such a sensitivity for children, but
our systematic analysis of children's open-ended verbal feedback did. It
showed that children can provide appropriate corrections to infelicitous
utterances containing logical connectives \emph{and} and \emph{or}.

\newpage

\section{References}\label{references}

\setlength{\parindent}{-0.5in} \setlength{\leftskip}{0.5in}

\hypertarget{refs}{}
\hypertarget{ref-Aloni2016}{}
Aloni, M. (2016). Disjunction. In E. N. Zalta (Ed.), \emph{The Stanford
encyclopedia of philosophy}. Stanford University. Retrieved from
\url{https://plato.stanford.edu/archives/win2016/entries/disjunction/}

\hypertarget{ref-barner2011accessing}{}
Barner, D., Brooks, N., \& Bale, A. (2011). Accessing the unsaid: The
role of scalar alternatives in children's pragmatic inference.
\emph{Cognition}, \emph{118}(1), 84--93.

\hypertarget{ref-braine1981development}{}
Braine, M. D., \& Rumain, B. (1981). Development of comprehension of
``or'': Evidence for a sequence of competencies. \emph{Journal of
Experimental Child Psychology}, \emph{31}(1), 46--70.

\hypertarget{ref-chierchia1998some}{}
Chierchia, G., Crain, S., Guasti, M. T., \& Thornton, R. (1998).
``Some'' and ``or'': A study on the emergence of logical form. In
\emph{Proceedings of the Boston University conference on language
development} (Vol. 22, pp. 97--108). Somerville, MA: Cascadilla Press.

\hypertarget{ref-chierchia2001acquisition}{}
Chierchia, G., Crain, S., Guasti, M. T., Gualmini, A., \& Meroni, L.
(2001). The acquisition of disjunction: Evidence for a grammatical view
of scalar implicatures. In \emph{Proceedings of the 25th Boston
University conference on language development} (pp. 157--168).
Somerville, MA: Cascadilla Press.

\hypertarget{ref-chierchia2004semantic}{}
Chierchia, G., Guasti, M. T., Gualmini, A., Meroni, L., Crain, S., \&
Foppolo, F. (2004). Semantic and pragmatic competence in children's and
adults' comprehension of or. In I. Noveck \& D. Sperber (Eds.),
\emph{Experimental pragmatics} (pp. 283--300). Basingstoke: Palgrave
Macmillan.

\hypertarget{ref-clark1973non}{}
Clark, E. V. (1973). Non-linguistic strategies and the acquisition of
word meanings. \emph{Cognition}, \emph{2}(2), 161--182.

\hypertarget{ref-crain2012emergence}{}
Crain, S. (2012). \emph{The emergence of meaning}. Cambridge: Cambridge
University Press.

\hypertarget{ref-crain2008logic}{}
Crain, S., \& Khlentzos, D. (2008). Is logic innate?
\emph{Biolinguistics}, \emph{2}(1), 024--056.

\hypertarget{ref-crain2010logic}{}
Crain, S., \& Khlentzos, D. (2010). The logic instinct. \emph{Mind \&
Language}, \emph{25}(1), 30--65.

\hypertarget{ref-crain1998investigations}{}
Crain, S., \& Thornton, R. (1998). \emph{Investigations in universal
grammar: A guide to experiments on the acquisition of syntax and
semantics}. Cambridge, MA: MIT Press.

\hypertarget{ref-crain2000acquisition}{}
Crain, S., Gualmini, A., \& Meroni, L. (2000). The acquisition of
logical words. \emph{LOGOS and Language}, \emph{1}, 49--59.

\hypertarget{ref-gazdar79}{}
Gazdar, G. (1979). \emph{Pragmatics: Implicature, presupposition, and
logical form}. New York: Academic Press.

\hypertarget{ref-goro2004acquisition}{}
Goro, T., \& Akiba, S. (2004). The acquisition of disjunction and
positive polarity in Japanese. In \emph{Proceedings of the 23rd West
Coast conference on formal linguistics} (pp. 251--264). Somerville, MA:
Cascadilla Press.

\hypertarget{ref-grice1989studies}{}
Grice, H. P. (1989). \emph{Studies in the way of words}. Cambridge, MA:
Harvard University Press.

\hypertarget{ref-gualminicrain2002}{}
Gualmini, A., \& Crain, S. (2002). Why no child or adult must learn de
Morgan's laws. In \emph{Proceedings of the Boston University conference
on language development}. Somerville, MA: Cascadilla Press.

\hypertarget{ref-gualmini2000}{}
Gualmini, A., Crain, S., \& Meroni, L. (2000). Acqisition of disjunction
in conditional sentences. In \emph{Proceedings of the boston university
conference on language development}.

\hypertarget{ref-gutzmann2014}{}
Gutzmann, D. (2014). Semantics vs. pragmatics. In L. Matthewson, C.
Meier, H. Rullmann, \& T. E. Zimmermann (Eds.), \emph{The companion to
semantics}. Oxford: Wiley.

\hypertarget{ref-horn1989natural}{}
Horn, L. (1989). \emph{A natural history of negation}. Chicago, IL:
University of Chicago Press.

\hypertarget{ref-horowitz2017trouble}{}
Horowitz, A. C., Schneider, R. M., \& Frank, M. C. (2017). The trouble
with quantifiers: Exploring children's deficits in scalar implicature.
\emph{Child Development}.

\hypertarget{ref-piaget1958growth}{}
Inhelder, B., \& Piaget, J. (1958). \emph{The growth of logical thinking
from childhood to adolescence: An essay on the construction of formal
operational structures} (Vol. 84). London: Routledge.

\hypertarget{ref-jasbiWaldonDegan2017}{}
Jasbi, M., Waldon, B., \& Degen, J. (submitted). \emph{Linking
hypothesis and number of response options modulate inferred scalar
implicature rate}.

\hypertarget{ref-johansson1975preschool}{}
Johansson, B. S., \& Sjolin, B. (1975). Preschool children's
understanding of the coordinators ``and'' and ``or''. \emph{Journal of
Experimental Child Psychology}, \emph{19}(2), 233--240.

\hypertarget{ref-katsos2014scalar}{}
Katsos, N. (2014). Scalar implicature. In D. Matthews (Ed.),
\emph{Pragmatic development in first language acquisition} (Vol. 10, p.
183---198). Amsterdam: John Benjamins.

\hypertarget{ref-katsos2011pragmatic}{}
Katsos, N., \& Bishop, D. V. (2011). Pragmatic tolerance: Implications
for the acquisition of informativeness and implicature.
\emph{Cognition}, \emph{120}(1), 67--81.

\hypertarget{ref-neimark1970}{}
Neimark, E. D. (1970). Development of comprehension of logical
connectives: Understanding of ``or''. \emph{Psychonomic Science},
\emph{21}(4), 217--219.

\hypertarget{ref-neimarkSlotnick1970}{}
Neimark, E. D., \& Slotnick, N. S. (1970). Development of the
understanding of logical connectives. \emph{Journal of Educational
Psychology}, \emph{61}(6p1), 451.

\hypertarget{ref-nitta1966basic}{}
Nitta, N., \& Nagano, S. (1966). Basic logical operations and their
verbal expressions: Child's conception of logical sum and product.
\emph{Research Bulletin of the National Institute for Educational
Research, Tokyo}, \emph{7}, 1--27.

\hypertarget{ref-notley2012notevery}{}
Notley, A., Thornton, R., \& Crain, S. (2012). English-speaking
children's interpretation of disjunction in the scope of ``not every''.
\emph{Biolinguistics}, \emph{6}(1), 32--69.

\hypertarget{ref-notley2012children}{}
Notley, A., Zhou, P., Jensen, B., \& Crain, S. (2012). Children's
interpretation of disjunction in the scope of ``before'': A comparison
of English and Mandarin. \emph{Journal of Child Language},
\emph{39}(03), 482--522.

\hypertarget{ref-noveck2001children}{}
Noveck, I. A. (2001). When children are more logical than adults:
Experimental investigations of scalar implicature. \emph{Cognition},
\emph{78}(2), 165--188.

\hypertarget{ref-paris1973comprehension}{}
Paris, S. G. (1973). Comprehension of language connectives and
propositional logical relationships. \emph{Journal of Experimental Child
Psychology}, \emph{16}(2), 278--291.

\hypertarget{ref-pouscoulous2009going}{}
Pouscoulous, N., \& Noveck, I. A. (2009). Going beyond semantics: The
development of pragmatic enrichment. In S. Foster-Cohen (Ed.),
\emph{Language acquisition} (pp. 196--215). Berlin: Springer.

\hypertarget{ref-pouscoulous2007developmental}{}
Pouscoulous, N., Noveck, I. A., Politzer, G., \& Bastide, A. (2007). A
developmental investigation of processing costs in implicature
production. \emph{Language Acquisition}, \emph{14}(4), 347--375.

\hypertarget{ref-reinhart2004processing}{}
Reinhart, T. (2004). The processing cost of reference set computation:
Acquisition of stress shift and focus. \emph{Language Acquisition},
\emph{12}(2), 109--155.

\hypertarget{ref-Singh2016}{}
Singh, R., Wexler, K., Astle-Rahim, A., Kamawar, D., \& Fox, D. (2016).
Children interpret disjunction as conjunction: Consequences for theories
of implicature and child development. \emph{Natural Language Semantics},
\emph{24}(4), 305--352.

\hypertarget{ref-skordosEtal2018}{}
Skordos, D., Feiman, R., Bale, A., \& Barner, D. (2018, July). \emph{Do
children interpret ``or'' conjunctively?} Retrieved from
\url{https://osf.io/2srxk/}

\hypertarget{ref-stiller2015ad}{}
Stiller, A. J., Goodman, N. D., \& Frank, M. C. (2015). Ad-hoc
implicature in preschool children. \emph{Language Learning and
Development}, \emph{11}(2), 176--190.

\hypertarget{ref-su2014acquisition}{}
Su, Y. (2014). The acquisition of logical connectives in child Mandarin.
\emph{Language Acquisition}, \emph{21}(2), 119--155.

\hypertarget{ref-su2013disjunction}{}
Su, Y., \& Crain, S. (2013). Disjunction and universal quantification in
child mandarin. \emph{Language and Linguistics}, \emph{14}(3), 599--631.

\hypertarget{ref-suppes1969young}{}
Suppes, P., \& Feldman, S. (1969). \emph{Young children's comprehension
of logical connectives.} \emph{ERIC}. Department of Health, Education,
Welfare. Office of Education.

\hypertarget{ref-tarski1941logic}{}
Tarski, A. (1941). \emph{Introduction to logic and to the methodology of
the deductive sciences}. Oxford University Press.

\hypertarget{ref-tieu2016}{}
Tieu, L., Yatsushiro, K., Cremers, A., Romoli, J., Sauerland, U., \&
Chemla, E. (2016). On the role of alternatives in the acquisition of
simple and complex disjunctions in french and japanese. \emph{Journal of
Semantics}.






\end{document}
